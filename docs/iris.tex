\documentclass[10pt]{article}
\usepackage{lmodern}
\usepackage[T1]{fontenc}
\usepackage[utf8]{inputenc}

\newif\ifslow\slowfalse %\slowtrue
\ifslow
	\usepackage[english]{babel}
	\usepackage[babel=true]{microtype}
\fi
\usepackage{geometry}

\usepackage[backend=biber]{biblatex}
\bibliography{bib}


\makeatletter%
\@ifundefined{basedir}{%
\newcommand\basedir{}%
}{}%
\makeatother%
%%%%%%%%%%%%%%%%%%%%%%%%%%%%%%%%%%%%%%%%%%%%%%%%%%%%%%%%%%%%%%%%
%% PACKAGES
%%%%%%%%%%%%%%%%%%%%%%%%%%%%%%%%%%%%%%%%%%%%%%%%%%%%%%%%%%%%%%%%

%\usepackage{amsmath}
\usepackage{amsfonts}
\usepackage{amsthm}
\usepackage{amssymb}
\usepackage{stmaryrd}

\usepackage{mathpartir}

\usepackage{\basedir pftools}
\usepackage{\basedir iris}
\usepackage{\basedir heaplang}

\usepackage{xcolor}  % for print version

\usepackage{graphicx}
\usepackage{enumitem}
\usepackage{semantic}
\usepackage{csquotes}

\usepackage{hyperref}

%%%%%%%%%%%%%%%%%%%%%%%%%%%%%%%%%%%%%%%%%%%%%%%%%%%%%%%%%%%%%%%%
%% SETUP
%%%%%%%%%%%%%%%%%%%%%%%%%%%%%%%%%%%%%%%%%%%%%%%%%%%%%%%%%%%%%%%%
\SetSymbolFont{stmry}{bold}{U}{stmry}{m}{n} % this fixes warnings when \boldsymbol is used with stmaryrd included

\extrarowheight=\jot	% else, arrays are scrunched compared to, say, aligned
\newcolumntype{.}{@{}}
% Array {rMcMl} modifies array {rcl}, putting mathrel-style spacing
% around the centered column. (We used this, for example, in laying
% out some of Iris' axioms. Generally, aligned is simpler but aligned
% does not work in mathpar because \\ inherits mathpar's 2em vskip.)
% The capital M stands for THICKMuskip. The smaller medmuskip would be
% right for mathbin-style spacing.
\newcolumntype{M}{@{\mskip\thickmuskip}}

\definecolor{StringRed}{rgb}{.637,0.082,0.082}
\definecolor{CommentGreen}{rgb}{0.0,0.55,0.3}
\definecolor{KeywordBlue}{rgb}{0.0,0.3,0.55}
\definecolor{LinkColor}{rgb}{0.55,0.0,0.3}
\definecolor{CiteColor}{rgb}{0.55,0.0,0.3}
\definecolor{HighlightColor}{rgb}{0.0,0.0,0.0}

\definecolor{grey}{rgb}{0.5,0.5,0.5}
\definecolor{red}{rgb}{1,0,0}

\hypersetup{%
  linktocpage=true, pdfstartview=FitV,
  breaklinks=true, pageanchor=true, pdfpagemode=UseOutlines,
  plainpages=false, bookmarksnumbered, bookmarksopen=true, bookmarksopenlevel=3,
  hypertexnames=true, pdfhighlight=/O,
  colorlinks=true,linkcolor=LinkColor,citecolor=CiteColor,
  urlcolor=LinkColor
}


%\theoremstyle{definition}
%\newtheorem{prop}{Prop}
\newtheorem{defn}{Definition}
\newtheorem{cor}{Corollary}
\newtheorem{conj}{Conj}
\newtheorem{lem}{Lemma}
\newtheorem{thm}{Theorem}

\newtheorem{exercise}{Exercise}

%%%%%%%%%%%%%%%%%%%%%%%%%%%%%%%%%%%%%%%%%%%%%%%%%%%%%%%%%%%%%%%%
%% GENERIC MACROS
%%%%%%%%%%%%%%%%%%%%%%%%%%%%%%%%%%%%%%%%%%%%%%%%%%%%%%%%%%%%%%%%
\newcommand*{\Sref}[1]{\hyperref[#1]{\S\ref*{#1}}}
\newcommand*{\secref}[1]{\hyperref[#1]{Section~\ref*{#1}}}
\newcommand*{\lemref}[1]{\hyperref[#1]{Lemma~\ref*{#1}}}
\newcommand*{\thmref}[1]{\hyperref[#1]{Theorem~\ref*{#1}}}
\newcommand{\corref}[1]{\hyperref[#1]{Cor.~\ref*{#1}}}
\newcommand*{\defref}[1]{\hyperref[#1]{Definition~\ref*{#1}}}
\newcommand*{\egref}[1]{\hyperref[#1]{Example~\ref*{#1}}}
\newcommand*{\appendixref}[1]{\hyperref[#1]{Appendix~\ref*{#1}}}
\newcommand*{\figref}[1]{\hyperref[#1]{Figure~\ref*{#1}}}
\newcommand*{\tabref}[1]{\hyperref[#1]{Table~\ref*{#1}}}

\newcommand{\changes}{{\bf\color{red}{Changes}}}
\newcommand{\TODO}{\vskip 4pt {\color{red}\bf TODO}}


\newcommand{\ie}{\emph{i.e.,} }
\newcommand{\cf}{\emph{c.f.} }
\newcommand{\eg}{\emph{e.g.,} }
\newcommand{\etal}{\emph{et~al.}}
\newcommand{\wrt}{w.r.t.~}

\newcommand{\aaron}[1]{{\color{red}\textbf{AT: #1}}}
\newcommand{\derek}[1]{{\color{red}\textbf{DD: #1}}}
\newcommand{\lars}[1]{{\color{red}\textbf{LB: #1}}}
\newcommand{\kasper}[1]{{\color{red}\textbf{KS: #1}}}
\newcommand{\ralf}[1]{{\color{red}\textbf{RJ: #1}}}
\newcommand{\dave}[1]{{\color{red}\textbf{PDS: #1}}}
\newcommand{\hush}[1]{}
\newcommand{\relaxguys}{%
	\let\aaron\hush%
	\let\derek\hush%
	\let\lars\hush%
	\let\kasper\hush%
	\let\ralf\hush%
	\let\dave\hush%
}



\title{\bfseries The Iris 3.1 Documentation}
\author{\url{http://plv.mpi-sws.org/iris/}}


\begin{document}

\maketitle
\thispagestyle{empty}
\vfill
\begin{abstract}
This document describes formally the Iris program logic.
Every result in this document has been fully verified in Coq.
The latest versions of this document and the Coq formalization can be found in the git repository at \url{https://gitlab.mpi-sws.org/FP/iris-coq/}.
For further information, visit the Iris project website at \url{http://plv.mpi-sws.org/iris/}.
\end{abstract}

\clearpage
\tableofcontents

\clearpage\begingroup
\section{Algebraic Structures}

\subsection{OFE}

The model of Iris lives in the category of \emph{Ordered Families of Equivalences} (OFEs).
This definition varies slightly from the original one in~\cite{catlogic}.

\begin{defn}
  An \emph{ordered family of equivalences} (OFE) is a tuple $(\ofe, ({\nequiv{n}} \subseteq \ofe \times \ofe)_{n \in \nat})$ satisfying
  \begin{align*}
    \All n. (\nequiv{n}) ~& \text{is an equivalence relation} \tagH{ofe-equiv} \\
    \All n, m.& n \geq m \Ra (\nequiv{n}) \subseteq (\nequiv{m}) \tagH{ofe-mono} \\
    \All x, y.& x = y \Lra (\All n. x \nequiv{n} y) \tagH{ofe-limit}
  \end{align*}
\end{defn}

The key intuition behind OFEs is that elements $x$ and $y$ are $n$-equivalent, notation $x \nequiv{n} y$, if they are \emph{equivalent for $n$ steps of computation}, \ie if they cannot be distinguished by a program running for no more than $n$ steps.
In other words, as $n$ increases, $\nequiv{n}$ becomes more and more refined (\ruleref{ofe-mono})---and in the limit, it agrees with plain equality (\ruleref{ofe-limit}).

\begin{defn}
  An element $x \in \ofe$ of an OFE is called \emph{discrete} if
  \[ \All y \in \ofe. x \nequiv{0} y \Ra x = y\]
  An OFE $A$ is called \emph{discrete} if all its elements are discrete.
  For a set $X$, we write $\Delta X$ for the discrete OFE with $x \nequiv{n} x' \eqdef x = x'$
\end{defn}

\begin{defn}
  A function $f : \ofe \to \ofeB$ between two OFEs is \emph{non-expansive} (written $f : \ofe \nfn \ofeB$) if
  \[\All n, x \in \ofe, y \in \ofe. x \nequiv{n} y \Ra f(x) \nequiv{n} f(y) \]
  It is \emph{contractive} if
  \[ \All n, x \in \ofe, y \in \ofe. (\All m < n. x \nequiv{m} y) \Ra f(x) \nequiv{n} f(y) \]
\end{defn}
Intuitively, applying a non-expansive function to some data will not suddenly introduce differences between seemingly equal data.
Elements that cannot be distinguished by programs within $n$ steps remain indistinguishable after applying $f$.

\begin{defn}
  The category $\OFEs$ consists of OFEs as objects, and non-expansive functions as arrows.
\end{defn}

Note that $\OFEs$ is bicartesian closed, \ie it has all sums, products and exponentials as well as an initial and a terminal object.
In particular:
\begin{defn}
  Given two OFEs $\ofe$ and $\ofeB$, the set of non-expansive functions $\set{f : \ofe \nfn \ofeB}$ is itself an OFE with
  \begin{align*}
    f \nequiv{n} g \eqdef{}& \All x \in \ofe. f(x) \nequiv{n} g(x)
  \end{align*}
\end{defn}

\begin{defn}
  A (bi)functor $F : \OFEs \to \OFEs$ is called \emph{locally non-expansive} if its action $F_1$ on arrows is itself a non-expansive map.
  Similarly, $F$ is called \emph{locally contractive} if $F_1$ is a contractive map.
\end{defn}
The function space $(-) \nfn (-)$ is a locally non-expansive bifunctor.
Note that the composition of non-expansive (bi)functors is non-expansive, and the composition of a non-expansive and a contractive (bi)functor is contractive.

One very important OFE is the OFE of \emph{step-indexed propositions}:
For every step-index, such a proposition either holds or does not hold.
Moreover, if a propositions holds for some $n$, it also has to hold for all smaller step-indices.
\begin{align*}
  \SProp \eqdef{}& \psetdown{\nat} \\
    \eqdef{}& \setComp{X \in \pset{\nat}}{ \All n, m. n \geq m \Ra n \in X \Ra m \in X } \\
  X \nequiv{n} Y \eqdef{}& \All m \leq n. m \in X \Lra m \in Y \\
  X \nincl{n} Y \eqdef{}& \All m \leq n. m \in X \Ra m \in Y
\end{align*}

\subsection{COFE}

COFEs are \emph{complete OFEs}, which means that we can take limits of arbitrary chains.

\begin{defn}[Chain]
  Given some set $\cofe$ and an indexed family $({\nequiv{n}} \subseteq \cofe \times \cofe)_{n \in \nat}$ of equivalence relations, a \emph{chain} is a function $c : \nat \to \cofe$ such that $\All n, m. n \leq m \Ra c (m) \nequiv{n} c (n)$.
\end{defn}

\begin{defn}
  A \emph{complete ordered family of equivalences} (COFE) is a tuple $(\cofe : \OFEs,  \lim : \chain(\cofe) \to \cofe)$ satisfying
  \begin{align*}
    \All n, c.& \lim(c) \nequiv{n} c(n) \tagH{cofe-compl}
  \end{align*}
\end{defn}

\begin{defn}
  The category $\COFEs$ consists of COFEs as objects, and non-expansive functions as arrows.
\end{defn}

The function space $\ofe \nfn \cofeB$ is a COFE if $\cofeB$ is a COFE (\ie the domain $\ofe$ can actually be just an OFE).
$\SProp$ as defined above is complete, \ie it is a COFE.

Completeness is necessary to take fixed-points.

\begin{thm}[Banach's fixed-point]
\label{thm:banach}
Given an inhabited COFE $\ofe$ and a contractive function $f : \ofe \to \ofe$, there exists a unique fixed-point $\fixp_T f$ such that $f(\fixp_T f) = \fixp_T f$.
Moreover, this theorem also holds if $f$ is just non-expansive, and $f^k$ is contractive for an arbitrary $k$.
\end{thm}

\begin{thm}[America and Rutten~\cite{America-Rutten:JCSS89,birkedal:metric-space}]
\label{thm:america_rutten}
Let $1$ be the discrete COFE on the unit type: $1 \eqdef \Delta \{ () \}$.
Given a locally contractive bifunctor $G : \COFEs^{\textrm{op}} \times \COFEs \to \COFEs$, and provided that \(G(1, 1)\) is inhabited,
then there exists a unique\footnote{Uniqueness is not proven in Coq.} COFE $\ofe$ such that $G(\ofe^{\textrm{op}}, \ofe) \cong \ofe$ (\ie the two are isomorphic in $\COFEs$).
\end{thm}

\subsection{RA}

\begin{defn}
  A \emph{resource algebra} (RA) is a tuple \\
  $(\monoid, \mvalFull :  \monoid \to \mProp, \mcore{{-}}:
  \monoid \to \maybe\monoid, (\mtimes) : \monoid \times \monoid \to \monoid)$ satisfying:
  \begin{align*}
    \All \melt, \meltB, \meltC.& (\melt \mtimes \meltB) \mtimes \meltC = \melt \mtimes (\meltB \mtimes \meltC) \tagH{ra-assoc} \\
    \All \melt, \meltB.& \melt \mtimes \meltB = \meltB \mtimes \melt \tagH{ra-comm} \\
    \All \melt.& \mcore\melt \in \monoid \Ra \mcore\melt \mtimes \melt = \melt \tagH{ra-core-id} \\
    \All \melt.& \mcore\melt \in \monoid \Ra \mcore{\mcore\melt} = \mcore\melt \tagH{ra-core-idem} \\
    \All \melt, \meltB.& \mcore\melt \in \monoid \land \melt \mincl \meltB \Ra \mcore\meltB \in \monoid \land \mcore\melt \mincl \mcore\meltB \tagH{ra-core-mono} \\
    \All \melt, \meltB.& \mvalFull(\melt \mtimes \meltB)  \Ra \mvalFull(\melt)  \tagH{ra-valid-op} \\
    \text{where}\qquad %\qquad\\
    \maybe\monoid \eqdef{}& \monoid \uplus \set{\mnocore} \qquad\qquad\qquad \melt^? \mtimes \mnocore \eqdef \mnocore \mtimes \melt^? \eqdef \melt^? \\
    \melt \mincl \meltB \eqdef{}& \Exists \meltC \in \monoid. \meltB = \melt \mtimes \meltC \tagH{ra-incl}
  \end{align*}
\end{defn}
Here, $\mProp$ is the set of (meta-level) propositions.
Think of \texttt{Prop} in Coq or $\mathbb{B}$ in classical mathematics.

RAs are closely related to \emph{Partial Commutative Monoids} (PCMs), with two key differences:
\begin{enumerate}
\item The composition operation on RAs is total (as opposed to the partial composition operation of a PCM), but there is a specific subset of \emph{valid} elements that is compatible with the composition operation (\ruleref{ra-valid-op}).
These valid elements are identified by the \emph{validity predicate} $\mvalFull$.

This take on partiality is necessary when defining the structure of \emph{higher-order} ghost state, \emph{cameras}, in the next subsection.

\item Instead of a single unit that is an identity to every element, we allow
for an arbitrary number of units, via a function $\mcore{{-}}$ assigning to an element $\melt$ its \emph{(duplicable) core} $\mcore\melt$, as demanded by \ruleref{ra-core-id}.
  We further demand that $\mcore{{-}}$ is idempotent (\ruleref{ra-core-idem}) and monotone (\ruleref{ra-core-mono}) with respect to the \emph{extension order}, defined similarly to that for PCMs (\ruleref{ra-incl}).

  Notice that the domain of the core is $\maybe\monoid$, a set that adds a dummy element $\mnocore$ to $\monoid$.
%  (This corresponds to the option type.)
  Thus, the core can be \emph{partial}: not all elements need to have a unit.
  We use the metavariable $\maybe\melt$ to indicate elements of  $\maybe\monoid$.
  We also lift the composition $(\mtimes)$ to $\maybe\monoid$.
  Partial cores help us to build interesting composite RAs from smaller primitives.

Notice also that the core of an RA is a strict generalization of the unit that any PCM must provide, since $\mcore{{-}}$ can always be picked as a constant function.
\end{enumerate}


\begin{defn}
  It is possible to do a \emph{frame-preserving update} from $\melt \in \monoid$ to $\meltsB \subseteq \monoid$, written $\melt \mupd \meltsB$, if
  \[ \All \maybe{\melt_\f} \in \maybe\monoid. \melt \mtimes \mvalFull(\maybe{\melt_\f}) \Ra \Exists \meltB \in \meltsB. \meltB \mtimes \mvalFull(\maybe{\melt_\f}) \]

  We further define $\melt \mupd \meltB \eqdef \melt \mupd \set\meltB$.
\end{defn}
The assertion $\melt \mupd \meltsB$ says that every element $\maybe{\melt_\f}$ compatible with $\melt$ (we also call such elements \emph{frames}), must also be compatible with some $\meltB \in \meltsB$.
Notice that $\maybe{\melt_\f}$ could be $\mnocore$, so the frame-preserving update can also be applied to elements that have \emph{no} frame.
Intuitively, this means that whatever assumptions the rest of the program is making about the state of $\gname$, if these assumptions are compatible with $\melt$, then updating to $\meltB$ will not invalidate any of these assumptions.
Since Iris ensures that the global ghost state is valid, this means that we can soundly update the ghost state from $\melt$ to a non-deterministically picked $\meltB \in \meltsB$.

\subsection{Cameras}

\begin{defn}
  A \emph{camera} is a tuple $(\monoid : \OFEs, \mval : \monoid \nfn \SProp, \mcore{{-}}: \monoid \nfn \maybe\monoid,\\ (\mtimes) : \monoid \times \monoid \nfn \monoid)$ satisfying:
  \begin{align*}
    \All \melt, \meltB, \meltC.& (\melt \mtimes \meltB) \mtimes \meltC = \melt \mtimes (\meltB \mtimes \meltC) \tagH{camera-assoc} \\
    \All \melt, \meltB.& \melt \mtimes \meltB = \meltB \mtimes \melt \tagH{camera-comm} \\
    \All \melt.& \mcore\melt \in \monoid \Ra \mcore\melt \mtimes \melt = \melt \tagH{camera-core-id} \\
    \All \melt.& \mcore\melt \in \monoid \Ra \mcore{\mcore\melt} = \mcore\melt \tagH{camera-core-idem} \\
    \All \melt, \meltB.& \mcore\melt \in \monoid \land \melt \mincl \meltB \Ra \mcore\meltB \in \monoid \land \mcore\melt \mincl \mcore\meltB \tagH{camera-core-mono} \\
    \All \melt, \meltB.& \mval(\melt \mtimes \meltB) \subseteq \mval(\melt)  \tagH{camera-valid-op} \\
    \All n, \melt, \meltB_1, \meltB_2.& \omit\rlap{$n \in \mval(\melt) \land \melt \nequiv{n} \meltB_1 \mtimes \meltB_2 \Ra {}$} \\
    &\Exists \meltC_1, \meltC_2. \melt = \meltC_1 \mtimes \meltC_2 \land \meltC_1 \nequiv{n} \meltB_1 \land \meltC_2 \nequiv{n} \meltB_2 \tagH{camera-extend} \\
    \text{where}\qquad\qquad\\
    \melt \mincl \meltB \eqdef{}& \Exists \meltC. \meltB = \melt \mtimes \meltC \tagH{camera-incl} \\
    \melt \mincl[n] \meltB \eqdef{}& \Exists \meltC. \meltB \nequiv{n} \melt \mtimes \meltC \tagH{camera-inclN}
  \end{align*}
\end{defn}

This is a natural generalization of RAs over OFEs.
All operations have to be non-expansive, and the validity predicate $\mval$ can now also depend on the step-index.
We define the plain $\mvalFull$ as the ``limit'' of the step-indexed approximation:
\[ \mvalFull(\melt) \eqdef \All n. n \in \mval(\melt) \]

\paragraph{The extension axiom (\ruleref{camera-extend}).}
Notice that the existential quantification in this axiom is \emph{constructive}, \ie it is a sigma type in Coq.
The purpose of this axiom is to compute $\melt_1$, $\melt_2$ completing the following square:

% RJ FIXME: Needs some magic to fix the baseline of the $\nequiv{n}$, or so
\begin{center}
\begin{tikzpicture}[every edge/.style={draw=none}]
  \node (a) at (0, 0) {$\melt$};
  \node (b) at (1.7, 0) {$\meltB$};
  \node (b12) at (1.7, -1) {$\meltB_1 \mtimes \meltB_2$};
  \node (a12) at (0, -1) {$\melt_1 \mtimes \melt_2$};

  \path (a) edge node {$\nequiv{n}$} (b);
  \path (a12) edge node {$\nequiv{n}$} (b12);
  \path (a) edge node [rotate=90] {$=$} (a12);
  \path (b) edge node [rotate=90] {$=$} (b12);
\end{tikzpicture}\end{center}
where the $n$-equivalence at the bottom is meant to apply to the pairs of elements, \ie we demand $\melt_1 \nequiv{n} \meltB_1$ and $\melt_2 \nequiv{n} \meltB_2$.
In other words, extension carries the decomposition of $\meltB$ into $\meltB_1$ and $\meltB_2$ over the $n$-equivalence of $\melt$ and $\meltB$, and yields a corresponding decomposition of $\melt$ into $\melt_1$ and $\melt_2$.
This operation is needed to prove that $\later$ commutes with separating conjunction:
\begin{mathpar}
  \axiom{\later (\prop * \propB) \Lra \later\prop * \later\propB}
\end{mathpar}

\begin{defn}
  An element $\munit$ of a camera $\monoid$ is called the \emph{unit} of $\monoid$ if it satisfies the following conditions:
  \begin{enumerate}[itemsep=0pt]
  \item $\munit$ is valid: \\ $\All n. n \in \mval(\munit)$
  \item $\munit$ is a left-identity of the operation: \\
    $\All \melt \in M. \munit \mtimes \melt = \melt$
  \item $\munit$ is its own core: \\ $\mcore\munit = \munit$
  \end{enumerate}
\end{defn}

\begin{lem}\label{lem:camera-unit-total-core}
  If $\monoid$ has a unit $\munit$, then the core $\mcore{{-}}$ is total, \ie $\All\melt. \mcore\melt \in \monoid$.
\end{lem}

\begin{defn}
  It is possible to do a \emph{frame-preserving update} from $\melt \in \monoid$ to $\meltsB \subseteq \monoid$, written $\melt \mupd \meltsB$, if
  \[ \All n, \maybe{\melt_\f}. n \in \mval(\melt \mtimes \maybe{\melt_\f}) \Ra \Exists \meltB \in \meltsB. n \in\mval(\meltB \mtimes \maybe{\melt_\f}) \]

  We further define $\melt \mupd \meltB \eqdef \melt \mupd \set\meltB$.
\end{defn}
Note that for RAs, this and the RA-based definition of a frame-preserving update coincide.

\begin{defn}
  A camera $\monoid$ is \emph{discrete} if it satisfies the following conditions:
  \begin{enumerate}[itemsep=0pt]
  \item $\monoid$ is a discrete COFE
  \item $\mval$ ignores the step-index: \\
    $\All \melt \in \monoid. 0 \in \mval(\melt) \Ra \All n. n \in \mval(\melt)$
  \end{enumerate}
\end{defn}
Note that every RA is a discrete camera, by picking the discrete COFE for the equivalence relation.
Furthermore, discrete cameras can be turned into RAs by ignoring their COFE structure, as well as the step-index of $\mval$.

\begin{defn}[Camera homomorphism]
  A function $f : \monoid_1 \to \monoid_2$ between two cameras is \emph{a camera homomorphism} if it satisfies the following conditions:
  \begin{enumerate}[itemsep=0pt]
  \item $f$ is non-expansive
  \item $f$ commutes with composition:\\
    $\All \melt_1 \in \monoid_1, \melt_2 \in \monoid_1. f(\melt_1) \mtimes f(\melt_2) = f(\melt_1 \mtimes \melt_2)$
  \item $f$ commutes with the core:\\
    $\All \melt \in \monoid_1. \mcore{f(\melt)} = f(\mcore{\melt})$
  \item $f$ preserves validity: \\
    $\All n, \melt \in \monoid_1. n \in \mval(\melt) \Ra n \in \mval(f(\melt))$
  \end{enumerate}
\end{defn}

\begin{defn}
  The category $\CMRAs$ consists of cameras as objects, and monotone functions as arrows.
\end{defn}
Note that every object/arrow in $\CMRAs$ is also an object/arrow of $\OFEs$.
The notion of a locally non-expansive (or contractive) bifunctor naturally generalizes to bifunctors between these categories.
%TODO: Discuss how we probably have a commuting square of functors between Set, RA, CMRA, COFE.

%%% Local Variables: 
%%% mode: latex
%%% TeX-master: "iris"
%%% End: 

\endgroup\clearpage\begingroup
\section{OFE and COFE Constructions}

\subsection{Trivial Pointwise Lifting}

The (C)OFE structure on many types can be easily obtained by pointwise lifting of the structure of the components.
This is what we do for option $\maybe\cofe$, product $(M_i)_{i \in I}$ (with $I$ some finite index set), sum $\cofe + \cofe'$ and finite partial functions $K \fpfn \monoid$ (with $K$ infinite countable).

\subsection{Next (Type-Level Later)}

Given an OFE $\cofe$, we define $\latert\cofe$ as follows (using a datatype-like notation to define the type):
\begin{align*}
  \latert\cofe \eqdef{}& \latertinj(x:\cofe) \\
  \latertinj(x) \nequiv{n} \latertinj(y) \eqdef{}& n = 0 \lor x \nequiv{n-1} y
\end{align*}
Note that in the definition of the carrier $\latert\cofe$, $\latertinj$ is a constructor (like the constructors in Coq), \ie this is short for $\setComp{\latertinj(x)}{x \in \cofe}$.

$\latert(-)$ is a locally \emph{contractive} functor from $\OFEs$ to $\OFEs$.


\subsection{Uniform Predicates}

Given a camera $\monoid$, we define the COFE $\UPred(\monoid)$ of \emph{uniform predicates} over $\monoid$ as follows:
\begin{align*}
\monoid \monnra \SProp \eqdef{}& \setComp{\pred: \monoid \nfn \SProp}
{\All n, \melt, \meltB. \melt \mincl[n] \meltB \Ra \pred(\melt) \nincl{n} \pred(\meltB)} \\
  \UPred(\monoid) \eqdef{}&  \faktor{\monoid \monnra \SProp}{\equiv} \\
  \pred \equiv \predB \eqdef{}& \All m, \melt. m \in \mval(\melt) \Ra (m \in \pred(\melt) \iff  m \in \predB(\melt)) \\
  \pred \nequiv{n} \predB \eqdef{}& \All m \le n, \melt. m \in \mval(\melt) \Ra (m \in \pred(\melt) \iff  m \in \predB(\melt))
\end{align*}
You can think of uniform predicates as monotone, step-indexed predicates over a camera that ``ignore'' invalid elements (as defined by the quotient).

$\UPred(-)$ is a locally non-expansive functor from $\CMRAs$ to $\COFEs$.

It is worth noting that the above quotient admits canonical
representatives. More precisely, one can show that every
equivalence class contains exactly one element $P_0$ such that:
\begin{align*}
  \All n, \melt.  (\mval(\melt) \nincl{n} P_0(\melt)) \Ra n \in P_0(\melt)  \tagH{UPred-canonical}
\end{align*}
Intuitively, this says that $P_0$ trivially holds whenever the resource is invalid.
Starting from any element $P$, one can find this canonical
representative by choosing $P_0(\melt) := \setComp{n}{n \in \mval(\melt) \Ra n \in P(\melt)}$.

Hence, as an alternative definition of $\UPred$, we could use the set
of canonical representatives. This alternative definition would
save us from using a quotient. However, the definitions of the various
connectives would get more complicated, because we have to make sure
they all verify \ruleref{UPred-canonical}, which sometimes requires some adjustments. We
would moreover need to prove one more property for every logical
connective.


\clearpage
\section{RA and Camera Constructions}

\subsection{Product}
\label{sec:prodm}

Given a family $(M_i)_{i \in I}$ of cameras ($I$ finite), we construct a camera for the product $\prod_{i \in I} M_i$ by lifting everything pointwise.

Frame-preserving updates on the $M_i$ lift to the product:
\begin{mathpar}
  \inferH{prod-update}
  {\melt \mupd_{M_i} \meltsB}
  {\mapinsert i \melt f \mupd \setComp{ \mapinsert i \meltB f}{\meltB \in \meltsB}}
\end{mathpar}

\subsection{Sum}
\label{sec:summ}

The \emph{sum camera} $\monoid_1 \csumm \monoid_2$ for any cameras $\monoid_1$ and $\monoid_2$ is defined as (again, we use a datatype-like notation):
\begin{align*}
  \monoid_1 \csumm \monoid_2 \eqdef{}& \cinl(\melt_1:\monoid_1) \mid \cinr(\melt_2:\monoid_2) \mid \mundef \\
  \mval(\mundef) \eqdef{}& \emptyset \\
  \mval(\cinl(\melt)) \eqdef{}& \mval_1(\melt)  \\
  \cinl(\melt_1) \mtimes \cinl(\meltB_1) \eqdef{}& \cinl(\melt_1 \mtimes \meltB_1)  \\
%  \munit \mtimes \ospending \eqdef{}& \ospending \mtimes \munit \eqdef \ospending \\
%  \munit \mtimes \osshot(\melt) \eqdef{}& \osshot(\melt) \mtimes \munit \eqdef \osshot(\melt) \\
  \mcore{\cinl(\melt_1)} \eqdef{}& \begin{cases}\mnocore & \text{if $\mcore{\melt_1} = \mnocore$} \\ \cinl({\mcore{\melt_1}}) & \text{otherwise} \end{cases}
\end{align*}
Above, $\mval_1$ refers to the validity of $\monoid_1$.
The validity, composition and core for $\cinr$ are defined symmetrically.
The remaining cases of the composition and core are all $\mundef$.

Notice that we added the artificial ``invalid'' (or ``undefined'') element $\mundef$ to this camera just in order to make certain compositions of elements (in this case, $\cinl$ and $\cinr$) invalid.

The step-indexed equivalence is inductively defined as follows:
\begin{mathpar}
  \infer{x \nequiv{n} y}{\cinl(x) \nequiv{n} \cinl(y)}

  \infer{x \nequiv{n} y}{\cinr(x) \nequiv{n} \cinr(y)}

  \axiom{\mundef \nequiv{n} \mundef}
\end{mathpar}


We obtain the following frame-preserving updates, as well as their symmetric counterparts:
\begin{mathpar}
  \inferH{sum-update}
  {\melt \mupd_{M_1} \meltsB}
  {\cinl(\melt) \mupd \setComp{ \cinl(\meltB)}{\meltB \in \meltsB}}

  \inferH{sum-swap}
  {\All \melt_\f \in M, n. n  \notin \mval(\melt \mtimes \melt_\f) \and \mvalFull(\meltB)}
  {\cinl(\melt) \mupd \cinr(\meltB)}
\end{mathpar}
Crucially, the second rule allows us to \emph{swap} the ``side'' of the sum that the camera is on if $\mval$ has \emph{no possible frame}.

\subsection{Option}

The definition of the camera/RA axioms already lifted the composition operation on $\monoid$ to one on $\maybe\monoid$.
We can easily extend this to a full camera by defining a suitable core, namely
\begin{align*}
  \mcore{\mnocore} \eqdef{}& \mnocore & \\
  \mcore{\maybe\melt} \eqdef{}& \mcore\melt & \text{If $\maybe\melt \neq \mnocore$}
\end{align*}
Notice that this core is total, as the result always lies in $\maybe\monoid$ (rather than in $\maybe{\mathord{\maybe\monoid}}$).

\subsection{Finite Partial Functions}
\label{sec:fpfnm}

Given some infinite countable $K$ and some camera $\monoid$, the set of finite partial functions $K \fpfn \monoid$ is equipped with a camera structure by lifting everything pointwise.

We obtain the following frame-preserving updates:
\begin{mathpar}
  \inferH{fpfn-alloc-strong}
  {\text{$G \subseteq K$ infinite} \and \mvalFull(\melt)}
  {\emptyset \mupd \setComp{\mapsingleton i \melt}{i \in G}}

  \inferH{fpfn-alloc}
  {\mvalFull(\melt)}
  {\emptyset \mupd \setComp{\mapsingleton i \melt}{i \in K}}

  \inferH{fpfn-update}
  {\melt \mupd_\monoid \meltsB}
  {\mapinsert i \melt f] \mupd \setComp{ \mapinsert i \meltB f}{\meltB \in \meltsB}}
\end{mathpar}
Above, $\mvalFull$ refers to the (full) validity of $\monoid$.

$K \fpfn (-)$ is a locally non-expansive functor from $\CMRAs$ to $\CMRAs$.

\subsection{Agreement}

Given some OFE $\cofe$, we define the camera $\agm(\cofe)$ as follows:
\begin{align*}
  \agm(\cofe) \eqdef{}& \setComp{\melt \in \finpset\cofe}{\melt \neq \emptyset} /\ {\sim} \\[-0.2em]
  \melt \nequiv{n} \meltB \eqdef{}& (\All x \in \melt. \Exists y \in \meltB. x \nequiv{n} y) \land (\All y \in \meltB. \Exists x \in \melt. x \nequiv{n} y) \\
  \textnormal{where }& \melt \sim \meltB \eqdef{} \All n. \melt \nequiv{n} \meltB  \\
~\\
%    \All n \in {\melt.V}.\, \melt.x \nequiv{n} \meltB.x \\
  \mval(\melt) \eqdef{}& \setComp{n}{ \All x, y \in \melt. x \nequiv{n} y } \\
  \mcore\melt \eqdef{}& \melt \\
  \melt \mtimes \meltB \eqdef{}& \melt \cup \meltB
\end{align*}
%Note that the carrier $\agm(\cofe)$ is a \emph{record} consisting of the two fields $c$ and $V$.

$\agm(-)$ is a locally non-expansive functor from $\OFEs$ to $\CMRAs$.

We define a non-expansive injection $\aginj$ into $\agm(\cofe)$ as follows:
\[ \aginj(x) \eqdef \set{x} \]
There are no interesting frame-preserving updates for $\agm(\cofe)$, but we can show the following:
\begin{mathpar}
  \axiomH{ag-val}{\mvalFull(\aginj(x))}

  \axiomH{ag-dup}{\aginj(x) = \aginj(x)\mtimes\aginj(x)}
  
  \axiomH{ag-agree}{n \in \mval(\aginj(x) \mtimes \aginj(y)) \Ra x \nequiv{n} y}
\end{mathpar}


\subsection{Exclusive Camera}

Given an OFE $\cofe$, we define a camera $\exm(\cofe)$ such that at most one $x \in \cofe$ can be owned:
\begin{align*}
  \exm(\cofe) \eqdef{}& \exinj(\cofe) \mid \mundef \\
  \mval(\melt) \eqdef{}& \setComp{n}{\melt \notnequiv{n} \mundef}
\end{align*}
All cases of composition go to $\mundef$.
\begin{align*}
  \mcore{\exinj(x)} \eqdef{}& \mnocore &
  \mcore{\mundef} \eqdef{}& \mundef
\end{align*}
Remember that $\mnocore$ is the ``dummy'' element in $\maybe\monoid$ indicating (in this case) that $\exinj(x)$ has no core.

The step-indexed equivalence is inductively defined as follows:
\begin{mathpar}
  \infer{x \nequiv{n} y}{\exinj(x) \nequiv{n} \exinj(y)}

  \axiom{\mundef \nequiv{n} \mundef}
\end{mathpar}
$\exm(-)$ is a locally non-expansive functor from $\OFEs$ to $\CMRAs$.

We obtain the following frame-preserving update:
\begin{mathpar}
  \inferH{ex-update}{}
  {\exinj(x) \mupd \exinj(y)}
\end{mathpar}

\subsection{Fractions}

We define an RA structure on the rational numbers in $(0, 1]$ as follows:
\begin{align*}
  \fracm \eqdef{}& \fracinj(\mathbb{Q} \cap (0, 1]) \mid \mundef \\
  \mvalFull(\melt) \eqdef{}& \melt \neq \mundef \\
  \fracinj(q_1) \mtimes \fracinj(q_2) \eqdef{}& \fracinj(q_1 + q_2) \quad \text{if $q_1 + q_2 \leq 1$} \\
  \mcore{\fracinj(x)} \eqdef{}& \bot \\
  \mcore{\mundef} \eqdef{}& \mundef
\end{align*}
All remaining cases of composition go to $\mundef$.
Frequently, we will write just $x$ instead of $\fracinj(x)$.

The most important property of this RA is that $1$ has no frame.
This is useful in combination with \ruleref{sum-swap}, and also when used with pairs:
\begin{mathpar}
  \inferH{pair-frac-change}{}
  {(1, a) \mupd (1, b)}
\end{mathpar}

%TODO: These need syncing with Coq
% \subsection{Finite Powerset Monoid}

% Given an infinite set $X$, we define a monoid $\textdom{PowFin}$ with carrier $\mathcal{P}^{\textrm{fin}}(X)$ as follows:
% \[
% \melt \cdot \meltB \;\eqdef\; \melt \cup \meltB \quad \mbox{if } \melt \cap \meltB = \emptyset
% \]

% We obtain:
% \begin{mathpar}
% 	\inferH{PowFinUpd}{}
% 		{\emptyset \mupd \{ \{x\} \mid x \in X  \}}
% \end{mathpar}

% \begin{proof}[Proof of \ruleref{PowFinUpd}]
% 	Assume some frame $\melt_\f \sep \emptyset$. Since $\melt_\f$ is finite and $X$ is infinite, there exists an $x \notin \melt_\f$.
% 	Pick that for the result.
% \end{proof}

% The powerset monoids is cancellative.
% \begin{proof}[Proof of cancellativity]
% 	Let $\melt_\f \mtimes \melt = \melt_\f \mtimes \meltB \neq \mzero$.
% 	So we have $\melt_\f \sep \melt$ and $\melt_\f \sep \meltB$, and we have to show $\melt = \meltB$.
% 	Assume $x \in \melt$. Hence $x \in \melt_\f \mtimes \melt$ and thus $x \in \melt_\f \mtimes \meltB$.
% 	By disjointness, $x \notin \melt_\f$ and hence $x \in meltB$.
% 	The other direction works the same way.
% \end{proof}



\subsection{Authoritative}
\label{sec:auth-camera}

Given a camera $M$, we construct $\authm(M)$ modeling someone owning an \emph{authoritative} element $\melt$ of $M$, and others potentially owning fragments $\meltB \mincl \melt$ of $\melt$.
We assume that $M$ has a unit $\munit$, and hence its core is total.
(If $M$ is an exclusive monoid, the construction is very similar to a half-ownership monoid with two asymmetric halves.)
\begin{align*}
\authm(M) \eqdef{}& \maybe{\exm(M)} \times M \\
\mval( (x, \meltB ) ) \eqdef{}& \setComp{ n }{ (x = \mnocore \land n \in \mval(\meltB)) \lor (\Exists \melt. x = \exinj(\melt) \land \meltB \mincl_n \melt \land n \in \mval(\melt)) } \\
  (x_1, \meltB_1) \mtimes (x_2, \meltB_2) \eqdef{}& (x_1 \mtimes x_2, \meltB_2 \mtimes \meltB_2) \\
  \mcore{(x, \meltB)} \eqdef{}& (\mnocore, \mcore\meltB) \\
  (x_1, \meltB_1) \nequiv{n} (x_2, \meltB_2) \eqdef{}& x_1 \nequiv{n} x_2 \land \meltB_1 \nequiv{n} \meltB_2
\end{align*}
Note that $(\mnocore, \munit)$ is the unit and asserts no ownership whatsoever, but $(\exinj(\munit), \munit)$ asserts that the authoritative element is $\munit$.

Let $\melt, \meltB \in M$.
We write $\authfull \melt$ for full ownership $(\exinj(\melt), \munit)$ and $\authfrag \meltB$ for fragmental ownership $(\mnocore, \meltB)$ and $\authfull \melt , \authfrag \meltB$ for combined ownership $(\exinj(\melt), \meltB)$.

The frame-preserving update involves the notion of a \emph{local update}:
\begin{defn}
  It is possible to do a \emph{local update} from $\melt_1$ and $\meltB_1$ to $\melt_2$ and $\meltB_2$, written $(\melt_1, \meltB_1) \lupd (\melt_2, \meltB_2)$, if
  \[ \All n, \maybe{\melt_\f}. n \in \mval(\melt_1) \land \melt_1 \nequiv{n} \meltB_1 \mtimes \maybe{\melt_\f} \Ra n \in \mval(\melt_2) \land \melt_2 \nequiv{n} \meltB_2 \mtimes \maybe{\melt_\f} \]
\end{defn}
In other words, the idea is that for every possible frame $\maybe{\melt_\f}$ completing $\meltB_1$ to $\melt_1$, the same frame also completes $\meltB_2$ to $\melt_2$.

We then obtain
\begin{mathpar}
  \inferH{auth-update}
  {(\melt_1, \meltB_1) \lupd (\melt_2, \meltB_2)}
  {\authfull \melt_1 , \authfrag \meltB_1 \mupd \authfull \melt_2 , \authfrag \meltB_2}
\end{mathpar}

\subsection{STS with Tokens}
\label{sec:sts-camera}

Given a state-transition system~(STS, \ie a directed graph) $(\STSS, {\stsstep} \subseteq \STSS \times \STSS)$, a set of tokens $\STST$, and a labeling $\STSL: \STSS \ra \wp(\STST)$ of \emph{protocol-owned} tokens for each state, we construct an RA modeling an authoritative current state and permitting transitions given a \emph{bound} on the current state and a set of \emph{locally-owned} tokens.

The construction follows the idea of STSs as described in CaReSL \cite{caresl}.
We first lift the transition relation to $\STSS \times \wp(\STST)$ (implementing a \emph{law of token conservation}) and define a stepping relation for the \emph{frame} of a given token set:
\begin{align*}
 (s, T) \stsstep (s', T') \eqdef{}& s \stsstep s' \land \STSL(s) \uplus T = \STSL(s') \uplus T' \\
 s \stsfstep{T} s' \eqdef{}& \Exists T_1, T_2. T_1 \disj \STSL(s) \cup T \land (s, T_1) \stsstep (s', T_2)
\end{align*}

We further define \emph{closed} sets of states (given a particular set of tokens) as well as the \emph{closure} of a set:
\begin{align*}
\STSclsd(S, T) \eqdef{}& \All s \in S. \STSL(s) \disj T \land \left(\All s'. s \stsfstep{T} s' \Ra s' \in S\right) \\
\upclose(S, T) \eqdef{}& \setComp{ s' \in \STSS}{\Exists s \in S. s \stsftrans{T} s' }
\end{align*}

The STS RA is defined as follows
\begin{align*}
  \monoid \eqdef{}& \STSauth(s:\STSS, T:\wp(\STST) \mid \STSL(s) \disj T) \mid{}\\& \STSfrag(S: \wp(\STSS), T: \wp(\STST) \mid \STSclsd(S, T) \land S \neq \emptyset) \mid \mundef \\
  \mvalFull(\melt) \eqdef{}& \melt \neq \mundef \\
  \STSfrag(S_1, T_1) \mtimes \STSfrag(S_2, T_2) \eqdef{}& \STSfrag(S_1 \cap S_2, T_1 \cup T_2) \qquad\qquad\qquad \text{if $T_1 \disj T_2$ and $S_1 \cap S_2 \neq \emptyset$} \\
  \STSfrag(S, T) \mtimes \STSauth(s, T') \eqdef{}& \STSauth(s, T') \mtimes \STSfrag(S, T) \eqdef \STSauth(s, T \cup T') \qquad \text{if $T \disj T'$ and $s \in S$} \\
  \mcore{\STSfrag(S, T)} \eqdef{}& \STSfrag(\upclose(S, \emptyset), \emptyset) \\
  \mcore{\STSauth(s, T)} \eqdef{}& \STSfrag(\upclose(\set{s}, \emptyset), \emptyset)
\end{align*}
The remaining cases are all $\mundef$.

We will need the following frame-preserving update:
\begin{mathpar}
  \inferH{sts-step}{(s, T) \ststrans (s', T')}
  {\STSauth(s, T) \mupd \STSauth(s', T')}

  \inferH{sts-weaken}
  {\STSclsd(S_2, T_2) \and S_1 \subseteq S_2 \and T_2 \subseteq T_1}
  {\STSfrag(S_1, T_1) \mupd \STSfrag(S_2, T_2)}
\end{mathpar}

\paragraph{The core is not a homomorphism.}
The core of the STS construction is only satisfying the RA axioms because we are \emph{not} demanding the core to be a homomorphism---all we demand is for the core to be monotone with respect the \ruleref{ra-incl}.

In other words, the following does \emph{not} hold for the STS core as defined above:
\[ \mcore\melt \mtimes \mcore\meltB = \mcore{\melt\mtimes\meltB} \]

To see why, consider the following STS:
\newcommand\st{\textlog{s}}
\newcommand\tok{\textlog{t}}
\begin{center}
  \begin{tikzpicture}[every node/.style=sts_state]
    \node at (0,0)   (s1) {$\st_1$};
    \node at (3,0)  (s2) {$\st_2$};
    \node at (9,0) (s3) {$\st_3$};
    \node at (6,0)  (s4) {$\st_4$\\$[\tok_1, \tok_2]$};
    
    \path[sts_arrows] (s2) edge  (s4);
    \path[sts_arrows] (s3) edge  (s4);
  \end{tikzpicture}
\end{center}
Now consider the following two elements of the STS RA:
\[ \melt \eqdef \STSfrag(\set{\st_1,\st_2}, \set{\tok_1}) \qquad\qquad
  \meltB \eqdef \STSfrag(\set{\st_1,\st_3}, \set{\tok_2}) \]

We have:
\begin{mathpar}
  {\melt\mtimes\meltB = \STSfrag(\set{\st_1}, \set{\tok_1, \tok_2})}

  {\mcore\melt = \STSfrag(\set{\st_1, \st_2, \st_4}, \emptyset)}

  {\mcore\meltB = \STSfrag(\set{\st_1, \st_3, \st_4}, \emptyset)}

  {\mcore\melt \mtimes \mcore\meltB = \STSfrag(\set{\st_1, \st_4}, \emptyset) \neq
    \mcore{\melt \mtimes \meltB} = \STSfrag(\set{\st_1}, \emptyset)}
\end{mathpar}

%%% Local Variables: 
%%% mode: latex
%%% TeX-master: "iris"
%%% End: 

\endgroup\clearpage\begingroup
\section{Base Logic}
\label{sec:base-logic}

The base logic is parameterized by an arbitrary camera $\monoid$ having a unit $\munit$.
By \lemref{lem:camera-unit-total-core}, this means that the core of $\monoid$ is a total function, so we will treat it as such in the following.
This defines the structure of resources that can be owned.

As usual for higher-order logics, you can furthermore pick a \emph{signature} $\Sig = (\SigType, \SigFn, \SigAx)$ to add more types, symbols and axioms to the language.
You have to make sure that $\SigType$ includes the base types:
\[
	\SigType \supseteq \{ \textlog{M}, \Prop \}
\]
Elements of $\SigType$ are ranged over by $\sigtype$.

Each function symbol in $\SigFn$ has an associated \emph{arity} comprising a natural number $n$ and an ordered list of $n+1$ types $\type$ (the grammar of $\type$ is defined below, and depends only on $\SigType$).
We write
\[
	\sigfn : \type_1, \dots, \type_n \to \type_{n+1} \in \SigFn
\]
to express that $\sigfn$ is a function symbol with the indicated arity.

Furthermore, $\SigAx$ is a set of \emph{axioms}, that is, terms $\term$ of type $\Prop$.
Again, the grammar of terms and their typing rules are defined below, and depends only on $\SigType$ and $\SigFn$, not on $\SigAx$.
Elements of $\SigAx$ are ranged over by $\sigax$.

\subsection{Grammar}\label{sec:grammar}

\paragraph{Syntax.}
Iris syntax is built up from a signature $\Sig$ and a countably infinite set $\Var$ of variables (ranged over by metavariables $\var$, $\varB$, $\varC$).
Below, $\melt$ ranges over $\monoid$ and $i$ ranges over $\set{1,2}$.

\begin{align*}
  \type \bnfdef{}&
      \sigtype \mid
      0 \mid
      1 \mid
      \type + \type \mid
      \type \times \type \mid
      \type \to \type
\\[0.4em]
  \term, \prop, \pred \bnfdef{}&
      \var \mid
      \sigfn(\term_1, \dots, \term_n) \mid
      \textlog{abort}\; \term \mid
      () \mid
      (\term, \term) \mid
      \pi_i\; \term \mid
      \Lam \var:\type.\term \mid
      \term(\term)  \mid
\\&
      \textlog{inj}_i\; \term \mid
      \textlog{match}\; \term \;\textlog{with}\; \Ret\textlog{inj}_1\; \var. \term \mid \Ret\textlog{inj}_2\; \var. \term \;\textlog{end} \mid
%
      \melt \mid
      \mcore\term \mid
      \term \mtimes \term \mid
\\&
    \FALSE \mid
    \TRUE \mid
    \term =_\type \term \mid
    \prop \Ra \prop \mid
    \prop \land \prop \mid
    \prop \lor \prop \mid
    \prop * \prop \mid
    \prop \wand \prop \mid
\\&
    \MU \var:\type. \term  \mid
    \Exists \var:\type. \prop \mid
    \All \var:\type. \prop \mid
%\\&
    \ownM{\term} \mid \mval(\term) \mid
    \always\prop \mid
    \plainly\prop \mid
    {\later\prop} \mid
    \upd \prop
\end{align*}
Well-typedness forces recursive definitions to be \emph{guarded}:
In $\MU \var. \term$, the variable $\var$ can only appear under the later $\later$ modality.
Furthermore, the type of the definition must be \emph{complete}.
The type $\Prop$ is complete, and if $\type$ is complete, then so is $\type' \to \type$.

Note that the modalities $\upd$, $\always$, $\plainly$ and $\later$ bind more tightly than $*$, $\wand$, $\land$, $\lor$, and $\Ra$.


\paragraph{Variable conventions.}
We assume that, if a term occurs multiple times in a rule, its free variables are exactly those binders which are available at every occurrence.


\subsection{Types}\label{sec:types}

Iris terms are simply-typed.
The judgment $\vctx \proves \wtt{\term}{\type}$ expresses that, in variable context $\vctx$, the term $\term$ has type $\type$.

A variable context, $\vctx = x_1:\type_1, \dots, x_n:\type_n$, declares a list of variables and their types.
In writing $\vctx, x:\type$, we presuppose that $x$ is not already declared in $\vctx$.

\judgment[Well-typed terms]{\vctx \proves_\Sig \wtt{\term}{\type}}
\begin{mathparpagebreakable}
%%% variables and function symbols
	\axiom{x : \type \proves \wtt{x}{\type}}
\and
	\infer{\vctx \proves \wtt{\term}{\type}}
		{\vctx, x:\type' \proves \wtt{\term}{\type}}
\and
	\infer{\vctx, x:\type', y:\type' \proves \wtt{\term}{\type}}
		{\vctx, x:\type' \proves \wtt{\term[x/y]}{\type}}
\and
	\infer{\vctx_1, x:\type', y:\type'', \vctx_2 \proves \wtt{\term}{\type}}
		{\vctx_1, x:\type'', y:\type', \vctx_2 \proves \wtt{\term[y/x,x/y]}{\type}}
\and
	\infer{
		\vctx \proves \wtt{\term_1}{\type_1} \and
		\cdots \and
		\vctx \proves \wtt{\term_n}{\type_n} \and
		\sigfn : \type_1, \dots, \type_n \to \type_{n+1} \in \SigFn
	}{
		\vctx \proves \wtt {\sigfn(\term_1, \dots, \term_n)} {\type_{n+1}}
	}
%%% empty, unit, products, sums
\and
	\infer{\vctx \proves \wtt\term{0}}
        {\vctx \proves \wtt{\textlog{abort}\; \term}\type}
\and
	\axiom{\vctx \proves \wtt{()}{1}}
\and
	\infer{\vctx \proves \wtt{\term}{\type_1} \and \vctx \proves \wtt{\termB}{\type_2}}
		{\vctx \proves \wtt{(\term,\termB)}{\type_1 \times \type_2}}
\and
	\infer{\vctx \proves \wtt{\term}{\type_1 \times \type_2} \and i \in \{1, 2\}}
		{\vctx \proves \wtt{\pi_i\,\term}{\type_i}}
\and
        \infer{\vctx \proves \wtt\term{\type_i} \and i \in \{1, 2\}}
        {\vctx \proves \wtt{\textlog{inj}_i\;\term}{\type_1 + \type_2}}
\and
        \infer{\vctx \proves \wtt\term{\type_1 + \type_2} \and
        \vctx, \var:\type_1 \proves \wtt{\term_1}\type \and
        \vctx, \varB:\type_2 \proves \wtt{\term_2}\type}
        {\vctx \proves \wtt{\textlog{match}\; \term \;\textlog{with}\; \Ret\textlog{inj}_1\; \var. \term_1 \mid \Ret\textlog{inj}_2\; \varB. \term_2 \;\textlog{end}}{\type}}
%%% functions
\and
	\infer{\vctx, x:\type \proves \wtt{\term}{\type'}}
		{\vctx \proves \wtt{\Lam x. \term}{\type \to \type'}}
\and
	\infer
	{\vctx \proves \wtt{\term}{\type \to \type'} \and \wtt{\termB}{\type}}
	{\vctx \proves \wtt{\term(\termB)}{\type'}}
%%% monoids
\and
        \infer{}{\vctx \proves \wtt\melt{\textlog{M}}}
\and
	\infer{\vctx \proves \wtt\melt{\textlog{M}}}{\vctx \proves \wtt{\mcore\melt}{\textlog{M}}}
\and
	\infer{\vctx \proves \wtt{\melt}{\textlog{M}} \and \vctx \proves \wtt{\meltB}{\textlog{M}}}
		{\vctx \proves \wtt{\melt \mtimes \meltB}{\textlog{M}}}
%%% props and predicates
\\
	\axiom{\vctx \proves \wtt{\FALSE}{\Prop}}
\and
	\axiom{\vctx \proves \wtt{\TRUE}{\Prop}}
\and
	\infer{\vctx \proves \wtt{\term}{\type} \and \vctx \proves \wtt{\termB}{\type}}
		{\vctx \proves \wtt{\term =_\type \termB}{\Prop}}
\and
	\infer{\vctx \proves \wtt{\prop}{\Prop} \and \vctx \proves \wtt{\propB}{\Prop}}
		{\vctx \proves \wtt{\prop \Ra \propB}{\Prop}}
\and
	\infer{\vctx \proves \wtt{\prop}{\Prop} \and \vctx \proves \wtt{\propB}{\Prop}}
		{\vctx \proves \wtt{\prop \land \propB}{\Prop}}
\and
	\infer{\vctx \proves \wtt{\prop}{\Prop} \and \vctx \proves \wtt{\propB}{\Prop}}
		{\vctx \proves \wtt{\prop \lor \propB}{\Prop}}
\and
	\infer{\vctx \proves \wtt{\prop}{\Prop} \and \vctx \proves \wtt{\propB}{\Prop}}
		{\vctx \proves \wtt{\prop * \propB}{\Prop}}
\and
	\infer{\vctx \proves \wtt{\prop}{\Prop} \and \vctx \proves \wtt{\propB}{\Prop}}
		{\vctx \proves \wtt{\prop \wand \propB}{\Prop}}
\and
	\infer{
		\vctx, \var:\type \proves \wtt{\term}{\type} \and
		\text{$\var$ is guarded in $\term$} \and
		\text{$\type$ is complete}
	}{
		\vctx \proves \wtt{\MU \var:\type. \term}{\type}
	}
\and
	\infer{\vctx, x:\type \proves \wtt{\prop}{\Prop}}
		{\vctx \proves \wtt{\Exists x:\type. \prop}{\Prop}}
\and
	\infer{\vctx, x:\type \proves \wtt{\prop}{\Prop}}
		{\vctx \proves \wtt{\All x:\type. \prop}{\Prop}}
\and
	\infer{\vctx \proves \wtt{\melt}{\textlog{M}}}
		{\vctx \proves \wtt{\ownM{\melt}}{\Prop}}
\and
	\infer{\vctx \proves \wtt{\melt}{\type} \and \text{$\type$ is a camera}}
		{\vctx \proves \wtt{\mval(\melt)}{\Prop}}
\and
	\infer{\vctx \proves \wtt{\prop}{\Prop}}
		{\vctx \proves \wtt{\always\prop}{\Prop}}
\and
	\infer{\vctx \proves \wtt{\prop}{\Prop}}
		{\vctx \proves \wtt{\plainly\prop}{\Prop}}
\and
	\infer{\vctx \proves \wtt{\prop}{\Prop}}
		{\vctx \proves \wtt{\later\prop}{\Prop}}
\and
	\infer{
		\vctx \proves \wtt{\prop}{\Prop}
	}{
		\vctx \proves \wtt{\upd \prop}{\Prop}
	}
\end{mathparpagebreakable}

\subsection{Proof Rules}
\label{sec:proof-rules}

The judgment $\vctx \mid \prop \proves \propB$ says that with free variables $\vctx$, proposition $\propB$ holds whenever assumption $\prop$ holds.
Most of the rules will entirely omit the variable contexts $\vctx$.
In this case, we assume the same arbitrary context is used for every constituent of the rules.
%Furthermore, an arbitrary \emph{boxed} assertion context $\always\pfctx$ may be added to every constituent.
Axioms $\vctx \mid \prop \provesIff \propB$ indicate that both $\vctx \mid \prop \proves \propB$ and $\vctx \mid \propB \proves \prop$ are proof rules of the logic.

\judgment{\vctx \mid \prop \proves \propB}
\paragraph{Laws of intuitionistic higher-order logic with equality.}
This is entirely standard.
\begin{mathparpagebreakable}
\infer[Asm]
  {}
  {\prop \proves \prop}
\and
\infer[Cut]
  {\prop \proves \propB \and \propB \proves \propC}
  {\prop \proves \propC}
\and
\infer[Eq]
  {\vctx,\var:\type \proves \wtt\propB\Prop \\ \vctx\mid\prop \proves \propB[\term/\var] \\ \vctx\mid\prop \proves \term =_\type \term'}
  {\vctx\mid\prop \proves \propB[\term'/\var]}
\and
\infer[Refl]
  {}
  {\TRUE \proves \term =_\type \term}
\and
\infer[$\bot$E]
  {}
  {\FALSE \proves \prop}
\and
\infer[$\top$I]
  {}
  {\prop \proves \TRUE}
\and
\infer[$\wedge$I]
  {\prop \proves \propB \\ \prop \proves \propC}
  {\prop \proves \propB \land \propC}
\and
\infer[$\wedge$EL]
  {\prop \proves \propB \land \propC}
  {\prop \proves \propB}
\and
\infer[$\wedge$ER]
  {\prop \proves \propB \land \propC}
  {\prop \proves \propC}
\and
\infer[$\vee$IL]
  {\prop \proves \propB }
  {\prop \proves \propB \lor \propC}
\and
\infer[$\vee$IR]
  {\prop \proves \propC}
  {\prop \proves \propB \lor \propC}
\and
\infer[$\vee$E]
  {\prop \proves \propC \\
   \propB \proves \propC}
  {\prop \lor \propB \proves \propC}
\and
\infer[$\Ra$I]
  {\prop \land \propB \proves \propC}
  {\prop \proves \propB \Ra \propC}
\and
\infer[$\Ra$E]
  {\prop \proves \propB \Ra \propC \\ \prop \proves \propB}
  {\prop \proves \propC}
\and
\infer[$\forall$I]
  { \vctx,\var : \type\mid\prop \proves \propB}
  {\vctx\mid\prop \proves \All \var: \type. \propB}
\and
\infer[$\forall$E]
  {\vctx\mid\prop \proves \All \var :\type. \propB \\
   \vctx \proves \wtt\term\type}
  {\vctx\mid\prop \proves \propB[\term/\var]}
\\
\infer[$\exists$I]
  {\vctx\mid\prop \proves \propB[\term/\var] \\
   \vctx \proves \wtt\term\type}
  {\vctx\mid\prop \proves \exists \var: \type. \propB}
\and
\infer[$\exists$E]
  {\vctx,\var : \type\mid\prop \proves \propB}
  {\vctx\mid\Exists \var: \type. \prop \proves \propB}
% \and
% \infer[$\lambda$]
%   {}
%   {\pfctx \proves (\Lam\var: \type. \prop)(\term) =_{\type\to\type'} \prop[\term/\var]}
% \and
% \infer[$\mu$]
%   {}
%   {\pfctx \proves \mu\var: \type. \prop =_{\type} \prop[\mu\var: \type. \prop/\var]}
\end{mathparpagebreakable}
Furthermore, we have the usual $\eta$ and $\beta$ laws for projections, $\textlog{abort}$, sum elimination, $\lambda$ and $\mu$.


\paragraph{Laws of (affine) bunched implications.}
\begin{mathpar}
\begin{array}{rMcMl}
  \TRUE * \prop &\provesIff& \prop \\
  \prop * \propB &\proves& \propB * \prop \\
  (\prop * \propB) * \propC &\proves& \prop * (\propB * \propC)
\end{array}
\and
\infer[$*$-mono]
  {\prop_1 \proves \propB_1 \and
   \prop_2 \proves \propB_2}
  {\prop_1 * \prop_2 \proves \propB_1 * \propB_2}
\and
\inferB[$\wand$I-E]
  {\prop * \propB \proves \propC}
  {\prop \proves \propB \wand \propC}
\end{mathpar}

\paragraph{Laws for the plainness modality.}
\begin{mathpar}
\infer[$\plainly$-mono]
  {\prop \proves \propB}
  {\plainly{\prop} \proves \plainly{\propB}}
\and
\infer[$\plainly$-E]{}
{\plainly\prop \proves \always\prop}
\and
\begin{array}[c]{rMcMl}
  (\plainly P \Ra \plainly Q) &\proves& \plainly (\plainly P \Ra Q) \\
\plainly ( ( P \Ra Q) \land (Q \Ra P ) ) &\proves& P =_{\Prop} Q
\end{array}
\and
\begin{array}[c]{rMcMl}
  \plainly{\prop} &\proves& \plainly\plainly\prop \\
  \All x. \plainly{\prop} &\proves& \plainly{\All x. \prop} \\
  \plainly{\Exists x. \prop} &\proves& \Exists x. \plainly{\prop}
\end{array}
%\and
%\infer[PropExt]{}{\plainly ( ( P \Ra Q) \land (Q \Ra P ) ) \proves P =_{\Prop} Q}
\end{mathpar}

\paragraph{Laws for the persistence modality.}
\begin{mathpar}
\infer[$\always$-mono]
  {\prop \proves \propB}
  {\always{\prop} \proves \always{\propB}}
\and
\infer[$\always$-E]{}
{\always\prop \proves \prop}
\and
\begin{array}[c]{rMcMl}
  (\plainly P \Ra \always Q) &\proves& \always (\plainly P \Ra Q) \\
  \always{\prop} \land \propB &\proves& \always{\prop} * \propB
\end{array}
\and
\begin{array}[c]{rMcMl}
  \always{\prop} &\proves& \always\always\prop \\
  \All x. \always{\prop} &\proves& \always{\All x. \prop} \\
  \always{\Exists x. \prop} &\proves& \Exists x. \always{\prop}
\end{array}
\end{mathpar}


\paragraph{Laws for the later modality.}
\begin{mathpar}
\infer[$\later$-mono]
  {\prop \proves \propB}
  {\later\prop \proves \later{\propB}}
\and
\inferhref{L{\"o}b}{Loeb}
  {}
  {(\later\prop\Ra\prop) \proves \prop}
\and
\begin{array}[c]{rMcMl}
  \All x. \later\prop &\proves& \later{\All x.\prop} \\
  \later\Exists x. \prop &\proves& \later\FALSE \lor {\Exists x.\later\prop}  \\
  \later\prop &\proves& \later\FALSE \lor (\later\FALSE \Ra \prop)
\end{array}
\and
\begin{array}[c]{rMcMl}
  \later{(\prop * \propB)} &\provesIff& \later\prop * \later\propB \\
  \always{\later\prop} &\provesIff& \later\always{\prop} \\
  \plainly{\later\prop} &\provesIff& \later\plainly{\prop}
\end{array}
\end{mathpar}


\paragraph{Laws for resources and validity.}
\begin{mathpar}
\begin{array}{rMcMl}
\ownM{\melt} * \ownM{\meltB} &\provesIff&  \ownM{\melt \mtimes \meltB} \\
\ownM\melt &\proves& \always{\ownM{\mcore\melt}} \\
\TRUE &\proves&  \ownM{\munit} \\
\later\ownM\melt &\proves& \Exists\meltB. \ownM\meltB \land \later(\melt = \meltB)
\end{array}
% \and
% \infer[valid-intro]
% {\melt \in \mval}
% {\TRUE \vdash \mval(\melt)}
% \and
% \infer[valid-elim]
% {\melt \notin \mval_0}
% {\mval(\melt) \proves \FALSE}
\and
\begin{array}{rMcMl}
\ownM{\melt} &\proves& \mval(\melt) \\
\mval(\melt \mtimes \meltB) &\proves& \mval(\melt) \\
\mval(\melt) &\proves& \always\mval(\melt)
\end{array}
\end{mathpar}


\paragraph{Laws for the basic update modality.}
\begin{mathpar}
\inferH{upd-mono}
{\prop \proves \propB}
{\upd\prop \proves \upd\propB}

\inferH{upd-intro}
{}{\prop \proves \upd \prop}

\inferH{upd-trans}
{}
{\upd \upd \prop \proves \upd \prop}

\inferH{upd-frame}
{}{\propB * \upd\prop \proves \upd (\propB * \prop)}

\inferH{upd-update}
{\melt \mupd \meltsB}
{\ownM\melt \proves \upd \Exists\meltB\in\meltsB. \ownM\meltB}

\inferH{upd-plainly}
{}
{\upd\plainly\prop \proves \prop}
\end{mathpar}
The premise in \ruleref{upd-update} is a \emph{meta-level} side-condition that has to be proven about $a$ and $B$.
%\ralf{Trouble is, we don't actually have $\in$ inside the logic...}

\subsection{Consistency}

The consistency statement of the logic reads as follows: For any $n$, we have
\begin{align*}
  \lnot(\TRUE \proves (\later)^n\spac\FALSE)
\end{align*}
where $(\later)^n$ is short for $\later$ being nested $n$ times.

The reason we want a stronger consistency than the usual $\lnot(\TRUE \proves \FALSE)$ is our modalities: it should be impossible to derive a contradiction below the modalities.
For $\always$ and $\plainly$, this follows from the elimination rules.
For updates, we use the fact that $\upd\FALSE \proves \upd\plainly\FALSE \proves \FALSE$.
However, there is no elimination rule for $\later$, so we declare that it is impossible to derive a contradiction below any number of laters.


%%% Local Variables:
%%% mode: latex
%%% TeX-master: "iris"
%%% End:

\endgroup\clearpage\begingroup
\section{Model and semantics}
\label{sec:model}

The semantics closely follows the ideas laid out in~\cite{catlogic}.

\paragraph{Semantic domains.}

The semantic  domains are interpreted as follows:
\[
\begin{array}[t]{@{}l@{\ }c@{\ }l@{}}
\Sem{\Prop} &\eqdef& \UPred(\monoid)  \\
\Sem{\textlog{M}} &\eqdef& \monoid \\
\Sem{0} &\eqdef& \Delta \emptyset \\
\Sem{1} &\eqdef& \Delta \{ () \}
\end{array}
\qquad\qquad
\begin{array}[t]{@{}l@{\ }c@{\ }l@{}}
\Sem{\type + \type'} &\eqdef& \Sem{\type} + \Sem{\type} \\
\Sem{\type \times \type'} &\eqdef& \Sem{\type} \times \Sem{\type} \\
\Sem{\type \to \type'} &\eqdef& \Sem{\type} \nfn \Sem{\type} \\
\end{array}
\]
For the remaining base types $\type$ defined by the signature $\Sig$, we pick an object $X_\type$ in $\OFEs$ and define
\[
\Sem{\type} \eqdef X_\type
\]
For each function symbol $\sigfn : \type_1, \dots, \type_n \to \type_{n+1} \in \SigFn$, we pick a function $\Sem{\sigfn} : \Sem{\type_1} \times \dots \times \Sem{\type_n} \nfn \Sem{\type_{n+1}}$.

\judgment[Interpretation of assertions.]{\Sem{\vctx \proves \term : \Prop} : \Sem{\vctx} \nfn \UPred(\monoid)}

Remember that $\UPred(\monoid)$ is isomorphic to $\monoid \monra \SProp$.
We are thus going to define the assertions as mapping CMRA elements to sets of step-indices.

\begin{align*}
	\Sem{\vctx \proves t =_\type u : \Prop}_\gamma &\eqdef
	\Lam \any. \setComp{n}{\Sem{\vctx \proves t : \type}_\gamma \nequiv{n} \Sem{\vctx \proves u : \type}_\gamma} \\
	\Sem{\vctx \proves \FALSE : \Prop}_\gamma &\eqdef \Lam \any. \emptyset \\
	\Sem{\vctx \proves \TRUE : \Prop}_\gamma &\eqdef \Lam \any. \nat \\
	\Sem{\vctx \proves \prop \land \propB : \Prop}_\gamma &\eqdef
	\Lam \melt. \Sem{\vctx \proves \prop : \Prop}_\gamma(\melt) \cap \Sem{\vctx \proves \propB : \Prop}_\gamma(\melt) \\
	\Sem{\vctx \proves \prop \lor \propB : \Prop}_\gamma &\eqdef
	\Lam \melt. \Sem{\vctx \proves \prop : \Prop}_\gamma(\melt) \cup \Sem{\vctx \proves \propB : \Prop}_\gamma(\melt) \\
	\Sem{\vctx \proves \prop \Ra \propB : \Prop}_\gamma &\eqdef
	\Lam \melt. \setComp{n}{\begin{aligned}
            \All m, \meltB.& m \leq n \land \melt \mincl \meltB \land m \in \mval(\meltB) \Ra {} \\
            & m \in \Sem{\vctx \proves \prop : \Prop}_\gamma(\meltB) \Ra {}\\& m \in \Sem{\vctx \proves \propB : \Prop}_\gamma(\meltB)\end{aligned}}\\
	\Sem{\vctx \proves \All \var : \type. \prop : \Prop}_\gamma &\eqdef
	\Lam \melt. \setComp{n}{ \All v \in \Sem{\type}. n \in \Sem{\vctx, \var : \type \proves \prop : \Prop}_{\mapinsert \var v \gamma}(\melt) } \\
	\Sem{\vctx \proves \Exists \var : \type. \prop : \Prop}_\gamma &\eqdef
        \Lam \melt. \setComp{n}{ \Exists v \in \Sem{\type}. n \in \Sem{\vctx, \var : \type \proves \prop : \Prop}_{\mapinsert \var v \gamma}(\melt) }
\end{align*}
\begin{align*}
	\Sem{\vctx \proves \prop * \propB : \Prop}_\gamma &\eqdef \Lam\melt. \setComp{n}{\begin{aligned}\Exists \meltB_1, \meltB_2. &\melt \nequiv{n} \meltB_1 \mtimes \meltB_2 \land {}\\& n \in \Sem{\vctx \proves \prop : \Prop}_\gamma(\meltB_1) \land n \in \Sem{\vctx \proves \propB : \Prop}_\gamma(\meltB_2)\end{aligned}}
\\
	\Sem{\vctx \proves \prop \wand \propB : \Prop}_\gamma &\eqdef
	\Lam \melt. \setComp{n}{\begin{aligned}
            \All m, \meltB.& m \leq n \land  m \in \mval(\melt\mtimes\meltB) \Ra {} \\
            & m \in \Sem{\vctx \proves \prop : \Prop}_\gamma(\meltB) \Ra {}\\& m \in \Sem{\vctx \proves \propB : \Prop}_\gamma(\melt\mtimes\meltB)\end{aligned}} \\
        \Sem{\vctx \proves \ownM{\term} : \Prop}_\gamma &\eqdef \Lam\meltB. \setComp{n}{\Sem{\vctx \proves \term : \textlog{M}}_\gamma \mincl[n] \meltB}  \\
        \Sem{\vctx \proves \mval(\term) : \Prop}_\gamma &\eqdef \Lam\any. \mval(\Sem{\vctx \proves \term : \textlog{M}}_\gamma) \\
	\Sem{\vctx \proves \always{\prop} : \Prop}_\gamma &\eqdef \Lam\melt. \Sem{\vctx \proves \prop : \Prop}_\gamma(\mcore\melt) \\
	\Sem{\vctx \proves \plainly{\prop} : \Prop}_\gamma &\eqdef \Lam\melt. \Sem{\vctx \proves \prop : \Prop}_\gamma(\munit) \\
	\Sem{\vctx \proves \later{\prop} : \Prop}_\gamma &\eqdef \Lam\melt. \setComp{n}{n = 0 \lor n-1 \in \Sem{\vctx \proves \prop : \Prop}_\gamma(\melt)}\\
        \Sem{\vctx \proves \upd\prop : \Prop}_\gamma &\eqdef \Lam\melt. \setComp{n}{\begin{aligned}
            \All m, \melt'. & m \leq n \land m \in \mval(\melt \mtimes \melt') \Ra {}\\& \Exists \meltB. m \in \mval(\meltB \mtimes \melt') \land m \in \Sem{\vctx \proves \prop :\Prop}_\gamma(\meltB)
          \end{aligned}
}
\end{align*}

For every definition, we have to show all the side-conditions: The maps have to be non-expansive and monotone.



\judgment[Interpretation of non-propositional terms]{\Sem{\vctx \proves \term : \type} : \Sem{\vctx} \nfn \Sem{\type}}
\begin{align*}
	\Sem{\vctx \proves x : \type}_\gamma &\eqdef \gamma(x) \\
	\Sem{\vctx \proves \sigfn(\term_1, \dots, \term_n) : \type_{n+1}}_\gamma &\eqdef \Sem{\sigfn}(\Sem{\vctx \proves \term_1 : \type_1}_\gamma, \dots, \Sem{\vctx \proves \term_n : \type_n}_\gamma) \\
	\Sem{\vctx \proves \Lam \var:\type. \term : \type \to \type'}_\gamma &\eqdef
	\Lam \termB : \Sem{\type}. \Sem{\vctx, \var : \type \proves \term : \type}_{\mapinsert \var \termB \gamma} \\
	\Sem{\vctx \proves \term(\termB) : \type'}_\gamma &\eqdef
	\Sem{\vctx \proves \term : \type \to \type'}_\gamma(\Sem{\vctx \proves \termB : \type}_\gamma) \\
	\Sem{\vctx \proves \MU \var:\type. \term : \type}_\gamma &\eqdef
	\mathit{fix}(\Lam \termB : \Sem{\type}. \Sem{\vctx, x : \type \proves \term : \type}_{\mapinsert \var \termB \gamma}) \\
  ~\\
	\Sem{\vctx \proves \textlog{abort}\;\term : \type}_\gamma &\eqdef \mathit{abort}_{\Sem\type}(\Sem{\vctx \proves \term:0}_\gamma) \\
	\Sem{\vctx \proves () : 1}_\gamma &\eqdef () \\
	\Sem{\vctx \proves (\term_1, \term_2) : \type_1 \times \type_2}_\gamma &\eqdef (\Sem{\vctx \proves \term_1 : \type_1}_\gamma, \Sem{\vctx \proves \term_2 : \type_2}_\gamma) \\
	\Sem{\vctx \proves \pi_i\; \term : \type_i}_\gamma &\eqdef \pi_i(\Sem{\vctx \proves \term : \type_1 \times \type_2}_\gamma) \\
        \Sem{\vctx \proves \textlog{inj}_i\;\term : \type_1 + \type_2}_\gamma &\eqdef \mathit{inj}_i(\Sem{\vctx \proves \term : \type_i}_\gamma) \\
        \Sem{\vctx \proves \textlog{match}\; \term \;\textlog{with}\; \Ret\textlog{inj}_1\; \var_1. \term_1 \mid \Ret\textlog{inj}_2\; \var_2. \term_2 \;\textlog{end} : \type }_\gamma &\eqdef 
    \Sem{\vctx, \var_i:\type_i \proves \term_i : \type}_{\mapinsert{\var_i}\termB \gamma} \\
    &\qquad \text{where $\Sem{\vctx \proves \term : \type_1 + \type_2}_\gamma = \mathit{inj}_i(\termB)$}
    \\
  ~\\
        \Sem{ \melt : \textlog{M} }_\gamma &\eqdef \melt \\
	\Sem{\vctx \proves \mcore\term : \textlog{M}}_\gamma &\eqdef \mcore{\Sem{\vctx \proves \term : \textlog{M}}_\gamma} \\
	\Sem{\vctx \proves \term \mtimes \termB : \textlog{M}}_\gamma &\eqdef
	\Sem{\vctx \proves \term : \textlog{M}}_\gamma \mtimes \Sem{\vctx \proves \termB : \textlog{M}}_\gamma
\end{align*}
%

An environment $\vctx$ is interpreted as the set of
finite partial functions $\rho$, with $\dom(\rho) = \dom(\vctx)$ and
$\rho(x)\in\Sem{\vctx(x)}$.
Above, $\mathit{fix}$ is the fixed-point on COFEs, and $\mathit{abort}_T$ is the unique function $\emptyset \to T$.

\paragraph{Logical entailment.}
We can now define \emph{semantic} logical entailment.

\typedsection{Interpretation of entailment}{\Sem{\vctx \mid \pfctx \proves \prop} : \mProp}

\[
\Sem{\vctx \mid \prop \proves \propB} \eqdef
\begin{aligned}[t]
\MoveEqLeft
\forall n \in \nat.\;
\forall \rs \in \monoid.\; 
\forall \gamma \in \Sem{\vctx},\;
\\&
n \in \Sem{\vctx \proves \prop : \Prop}_\gamma(\rs)
\Ra n \in \Sem{\vctx \proves \propB : \Prop}_\gamma(\rs)
\end{aligned}
\]

The following soundness theorem connects syntactic and semantic entailment.
It is proven by showing that all the syntactic proof rules of \Sref{sec:base-logic} can be validated in the model.
\[ \vctx \mid \prop \proves \propB \Ra \Sem{\vctx \mid \prop \proves \propB} \]

It now becomes straight-forward to show consistency of the logic.

%%% Local Variables:
%%% mode: latex
%%% TeX-master: "iris"
%%% End:

\endgroup\clearpage\begingroup
\section{Extensions of the Base Logic}

In this section we discuss some additional constructions that we define within and on top of the base logic.
These are not ``extensions'' in the sense that they change the proof power of the logic, they just form useful derived principles.

\subsection{Derived Rules about Base Connectives}
We collect here some important and frequently used derived proof rules.
\begin{mathparpagebreakable}
  \inferhref{L{\"o}b}{Loeb}
  {}
  {(\later\prop\Ra\prop) \proves \prop}

  \infer{}
  {\prop \Ra \propB \proves \prop \wand \propB}

  \infer{}
  {\prop * \Exists\var.\propB \provesIff \Exists\var. \prop * \propB}

  \infer{}
  {\prop * \All\var.\propB \proves \All\var. \prop * \propB}

  \infer{}
  {\always(\prop*\propB) \provesIff \always\prop * \always\propB}

  \infer{}
  {\always(\prop \Ra \propB) \proves \always\prop \Ra \always\propB}

  \infer{}
  {\always(\prop \wand \propB) \proves \always\prop \wand \always\propB}

  \infer{}
  {\always(\prop \wand \propB) \provesIff \always(\prop \Ra \propB)}

  \infer{}
  {\later(\prop \Ra \propB) \proves \later\prop \Ra \later\propB}

  \infer{}
  {\later(\prop \wand \propB) \proves \later\prop \wand \later\propB}

  \infer{}
  {\TRUE \proves \plainly\TRUE}
\end{mathparpagebreakable}

Noteworthy here is the fact that Löb induction can be derived from $\later$-introduction and the fact that we can take fixed-points of functions where the recursive occurrences are below $\later$.%
\footnote{See \url{https://en.wikipedia.org/wiki/L\%C3\%B6b\%27s_theorem}.}
Furthermore, $\TRUE \proves \plainly\TRUE$ can be derived via $\plainly$ commuting with universal quantification ranging over the empty type $0$.
To derive that existential quantifiers commute with separating conjunction requires an intermediate step using a magic wand: From $P * \exists x, Q \vdash \Exists x. P * Q$ we can deduce $\Exists x. Q \vdash P \wand \Exists x. P * Q$ and then proceed via $\exists$-elimination.

\subsection{Persistent Propositions}
We call a proposition $\prop$ \emph{persistent} if $\prop \proves \always\prop$.
These are propositions that ``do not own anything'', so we can (and will) treat them like ``normal'' intuitionistic propositions.

Of course, $\always\prop$ is persistent for any $\prop$.
Furthermore, by the proof rules given in \Sref{sec:proof-rules}, $\TRUE$, $\FALSE$, $t = t'$ as well as $\ownGhost\gname{\mcore\melt}$ and $\mval(\melt)$ are persistent.
Persistence is preserved by conjunction, disjunction, separating conjunction as well as universal and existential quantification and $\later$.



\subsection{Timeless Propositions and Except-0}

One of the troubles of working in a step-indexed logic is the ``later'' modality $\later$.
It turns out that we can somewhat mitigate this trouble by working below the following \emph{except-0} modality:
\[ \diamond \prop \eqdef \later\FALSE \lor \prop \]
Except-0 satisfies the usual laws of a ``monadic'' modality (similar to, \eg the update modalities):
\begin{mathpar}
  \inferH{ex0-mono}
  {\prop \proves \propB}
  {\diamond\prop \proves \diamond\propB}

  \axiomH{ex0-intro}
  {\prop \proves \diamond\prop}

  \axiomH{ex0-idem}
  {\diamond\diamond\prop \proves \diamond\prop}

\begin{array}[c]{rMcMl}
  \diamond{(\prop * \propB)} &\provesIff& \diamond\prop * \diamond\propB \\
  \diamond{(\prop \land \propB)} &\provesIff& \diamond\prop \land \diamond\propB \\
  \diamond{(\prop \lor \propB)} &\provesIff& \diamond\prop \lor \diamond\propB
\end{array}

\begin{array}[c]{rMcMl}
  \diamond{\All x. \prop} &\provesIff& \All x. \diamond{\prop}   \\
  \diamond{\Exists x. \prop} &\provesIff& \Exists x. \diamond{\prop} \\
  \diamond\always{\prop} &\provesIff& \always\diamond{\prop} \\
  \diamond\later\prop &\proves& \later{\prop}
\end{array}
\end{mathpar}
In particular, from \ruleref{ex0-mono} and \ruleref{ex0-idem} we can derive a ``bind''-like elimination rule:
\begin{mathpar}
  \inferH{ex0-elim}
  {\prop \proves \diamond\propB}
  {\diamond\prop \proves \diamond\propB}
\end{mathpar}

This modality is useful because there is a class of propositions which we call \emph{timeless} propositions, for which we have
\[ \timeless{\prop} \eqdef \later\prop \proves \diamond\prop  \]
In other words, when working below the except-0 modality, we can \emph{strip
  away} the later from timeless propositions (using \ruleref{ex0-elim}):
\begin{mathpar}
  \inferH{ex0-timeless-strip}{\timeless{\prop} \and \prop \proves \diamond\propB}
  {\later\prop \proves \diamond\propB}
\end{mathpar}

 In fact, it turns out that we can strip away later from timeless propositions even when working under the later modality:
\begin{mathpar}
  \inferH{later-timeless-strip}{\timeless{\prop} \and \prop \proves \later \propB}
  {\later\prop \proves \later\propB}
\end{mathpar}
This follows from $\later \prop \proves \later\FALSE \lor \prop$, and then by straightforward disjunction elimination.

The following rules identify the class of timeless propositions:
\begin{mathparpagebreakable}
\infer
{\vctx \proves \timeless{\prop} \and \vctx \proves \timeless{\propB}}
{\vctx \proves \timeless{\prop \land \propB}}

\infer
{\vctx \proves \timeless{\prop} \and \vctx \proves \timeless{\propB}}
{\vctx \proves \timeless{\prop \lor \propB}}

\infer
{\vctx \proves \timeless{\prop} \and \vctx \proves \timeless{\propB}}
{\vctx \proves \timeless{\prop * \propB}}

\infer
{\vctx \proves \timeless{\prop}}
{\vctx \proves \timeless{\always\prop}}

\infer
{\vctx \proves \timeless{\propB}}
{\vctx \proves \timeless{\prop \Ra \propB}}

\infer
{\vctx \proves \timeless{\propB}}
{\vctx \proves \timeless{\prop \wand \propB}}

\infer
{\vctx,\var:\type \proves \timeless{\prop}}
{\vctx \proves \timeless{\All\var:\type.\prop}}

\infer
{\vctx,\var:\type \proves \timeless{\prop}}
{\vctx \proves \timeless{\Exists\var:\type.\prop}}

\axiom{\timeless{\TRUE}}

\axiom{\timeless{\FALSE}}

\infer
{\text{$\term$ or $\term'$ is a discrete OFE element}}
{\timeless{\term =_\type \term'}}

\infer
{\text{$\melt$ is a discrete OFE element}}
{\timeless{\ownM\melt}}

\infer
{\text{$\melt$ is an element of a discrete camera}}
{\timeless{\mval(\melt)}}
\end{mathparpagebreakable}


\subsection{Dynamic Composeable Higher-Order Resources}
\label{sec:composeable-resources}

The base logic described in \Sref{sec:base-logic} works over an arbitrary camera $\monoid$ defining the structure of the resources.
It turns out that we can generalize this further and permit picking cameras ``$\iFunc(\Prop)$'' that depend on the structure of propositions themselves.
Of course, $\Prop$ is just the syntactic type of propositions; for this to make sense we have to look at the semantics.

Furthermore, there is a composability problem with the given logic: if we have one proof performed with camera $\monoid_1$, and another proof carried out with a \emph{different} camera $\monoid_2$, then the two proofs are actually carried out in two \emph{entirely separate logics} and hence cannot be combined.

Finally, in many cases just having a single ``instance'' of a camera available for reasoning is not enough.
For example, when reasoning about a dynamically allocated data structure, every time a new instance of that data structure is created, we will want a fresh resource governing the state of this particular instance.
While it would be possible to handle this problem whenever it comes up, it turns out to be useful to provide a general solution.

The purpose of this section is to describe how we solve these issues.

\paragraph{Picking the resources.}
The key ingredient that we will employ on top of the base logic is to give some more fixed structure to the resources.
To instantiate the logic with dynamic higher-order ghost state, the user picks a family of locally contractive bifunctors $(\iFunc_i : \COFEs^\op \times \COFEs \to \CMRAs)_{i \in \mathcal{I}}$.
(This is in contrast to the base logic, where the user picks a single, fixed camera that has a unit.)

From this, we construct the bifunctor defining the overall resources as follows:
\begin{align*}
  \GName \eqdef{}& \nat \\
  \textdom{ResF}(\ofe^\op, \ofe) \eqdef{}& \prod_{i \in \mathcal I} \GName \fpfn \iFunc_i(\ofe^\op, \ofe)
\end{align*}
We will motivate both the use of a product and the finite partial function below.
$\textdom{ResF}(\ofe^\op, \ofe)$ is a camera by lifting the individual cameras pointwise, and it has a unit (using the empty finite partial function).
Furthermore, since the $\iFunc_i$ are locally contractive, so is $\textdom{ResF}$.

Now we can write down the recursive domain equation:
\[ \iPreProp \cong \UPred(\textdom{ResF}(\iPreProp, \iPreProp)) \]
Here, $\iPreProp$ is a COFE defined as the fixed-point of a locally contractive bifunctor, which exists and is unique up to isomorphism by \thmref{thm:america_rutten}, so we obtain some object $\iPreProp$ such that:
\begin{align*}
  \Res &\eqdef \textdom{ResF}(\iPreProp, \iPreProp) \\
  \iProp &\eqdef \UPred(\Res) \\
	\wIso &: \iProp \nfn \iPreProp \\
	\wIso^{-1} &: \iPreProp \nfn \iProp \\
  \wIso(\wIso^{-1}(x)) &\eqdef x \\
  \wIso^{-1}(\wIso(x)) &\eqdef x
\end{align*}
Now we can instantiate the base logic described in \Sref{sec:base-logic} with $\Res$ as the chosen camera:
\[ \Sem{\Prop} \eqdef \UPred(\Res) \]
We obtain that $\Sem{\Prop} = \iProp$.
Effectively, we just defined a way to instantiate the base logic with $\Res$ as the camera of resources, while providing a way for $\Res$ to depend on $\iPreProp$, which is isomorphic to $\Sem\Prop$.

We thus obtain all the rules of \Sref{sec:base-logic}, and furthermore, we can use the maps $\wIso$ and $\wIso^{-1}$ \emph{in the logic} to convert between logical propositions $\Sem\Prop$ and the domain $\iPreProp$ which is used in the construction of $\Res$ -- so from elements of $\iPreProp$, we can construct elements of $\Sem{\textlog M}$, which are the elements that can be owned in our logic.

\paragraph{Proof composability.}
To make our proofs composeable, we \emph{generalize} our proofs over the family of functors.
This is possible because we made $\Res$ a \emph{product} of all the cameras picked by the user, and because we can actually work with that product ``pointwise''.
So instead of picking a \emph{concrete} family, proofs will assume to be given an \emph{arbitrary} family of functors, plus a proof that this family \emph{contains the functors they need}.
Composing two proofs is then merely a matter of conjoining the assumptions they make about the functors.
Since the logic is entirely parametric in the choice of functors, there is no trouble reasoning without full knowledge of the family of functors.

Only when the top-level proof is completed we will ``close'' the proof by picking a concrete family that contains exactly those functors the proof needs.

\paragraph{Dynamic resources.}
Finally, the use of finite partial functions lets us have as many instances of any camera as we could wish for:
Because there can only ever be finitely many instances already allocated, it is always possible to create a fresh instance with any desired (valid) starting state.
This is best demonstrated by giving some proof rules.

So let us first define the notion of ghost ownership that we use in this logic.
Assuming that the family of functors contains the functor $\Sigma_i$ at index $i$, and furthermore assuming that $\monoid_i = \Sigma_i(\iPreProp, \iPreProp)$, given some $\melt \in \monoid_i$ we define:
\[ \ownGhost\gname{\melt:\monoid_i} \eqdef \ownM{(\ldots, \emptyset, i:\mapsingleton \gname \melt, \emptyset, \ldots)} \]
This is ownership of the pair (element of the product over all the functors) that has the empty finite partial function in all components \emph{except for} the component corresponding to index $i$, where we own the element $\melt$ at index $\gname$ in the finite partial function.

We can show the following properties for this form of ownership:
\begin{mathparpagebreakable}
  \inferH{res-alloc}{\text{$G$ infinite} \and \melt \in \mval_{M_i}}
  {  \TRUE \proves \upd \Exists\gname\in G. \ownGhost\gname{\melt : M_i}
  }
  \and
  \inferH{res-update}
    {\melt \mupd_{M_i} B}
    {\ownGhost\gname{\melt : M_i} \proves \upd \Exists \meltB\in B. \ownGhost\gname{\meltB : M_i}}

  \inferH{res-empty}
  {\text{$\munit$ is a unit of $M_i$}}
  {\TRUE \proves \upd \ownGhost\gname\munit}

  \axiomH{res-op}
    {\ownGhost\gname{\melt : M_i} * \ownGhost\gname{\meltB : M_i} \provesIff \ownGhost\gname{\melt\mtimes\meltB : M_i}}

  \axiomH{res-valid}
    {\ownGhost\gname{\melt : M_i} \Ra \mval_{M_i}(\melt)}

  \inferH{res-timeless}
    {\text{$\melt$ is a discrete OFE element}}
    {\timeless{\ownGhost\gname{\melt : M_i}}}
\end{mathparpagebreakable}

Below, we will always work within (an instance of) the logic as described here.
Whenever a camera is used in a proof, we implicitly assume it to be available in the global family of functors.
We will typically leave the $M_i$ implicit when asserting ghost ownership, as the type of $\melt$ will be clear from the context.





%%% Local Variables:
%%% mode: latex
%%% TeX-master: "iris"
%%% End:

\endgroup\clearpage\begingroup
\section{Language}
\label{sec:language}

A \emph{language} $\Lang$ consists of a set \Expr{} of \emph{expressions} (metavariable $\expr$), a set \Val{} of \emph{values} (metavariable $\val$), a set $\Obs$ of \emph{observations}\footnote{See \url{https://gitlab.mpi-sws.org/iris/iris/merge_requests/173} for how observations are useful to encode prophecy variables.} (or ``observable events'') and a set $\State$ of \emph{states} (metavariable $\state$) such that
\begin{itemize}[itemsep=0pt]
\item There exist functions $\ofval : \Val \to \Expr$ and $\toval : \Expr \pfn \Val$ (notice the latter is partial), such that
\begin{mathpar}
{\All \expr, \val. \toval(\expr) = \val \Ra \ofval(\val) = \expr} \and
{\All\val. \toval(\ofval(\val)) = \val} 
\end{mathpar}
\item There exists a \emph{primitive reduction relation} \[(-,- \;\step[-]\; -,-,-) \subseteq (\Expr \times \State) \times \List(\Obs) \times (\Expr \times \State \times \List(\Expr))\]
  A reduction $\expr_1, \state_1 \step[\vec\obs] \expr_2, \state_2, \vec\expr$ indicates that, when $\expr_1$ in state $\state_1$ reduces to $\expr_2$ with new state $\state_2$, the new threads in the list $\vec\expr$ is forked off and the observations $\vec\obs$ are made.
  We will write $\expr_1, \state_1 \step \expr_2, \state_2$ for $\expr_1, \state_1 \step[()] \expr_2, \state_2, ()$, \ie when no threads are forked off and no observations are made. \\
\item All values are stuck:
\[ \expr, \_ \step  \_, \_, \_ \Ra \toval(\expr) = \bot \]
\end{itemize}

\begin{defn}
  An expression $\expr$ and state $\state$ are \emph{reducible} (written $\red(\expr, \state)$) if
  \[ \Exists \vec\obs, \expr_2, \state_2, \vec\expr. \expr,\state \step[\vec\obs] \expr_2,\state_2,\vec\expr \]
\end{defn}

\begin{defn}
  An expression $\expr$ is \emph{weakly atomic} if it reduces in one step to something irreducible:
  \[ \atomic(\expr) \eqdef \All\state_1, \vec\obs, \expr_2, \state_2, \vec\expr. \expr, \state_1 \step[\vec\obs] \expr_2, \state_2, \vec\expr \Ra \lnot \red(\expr_2, \state_2) \]
  It is \emph{strongly atomic} if it reduces in one step to a value:
  \[ \stronglyAtomic(\expr) \eqdef \All\state_1, \vec\obs, \expr_2, \state_2, \vec\expr. \expr, \state_1 \step[\vec\obs] \expr_2, \state_2, \vec\expr \Ra \toval(\expr_2) \neq \bot \]
\end{defn}
We need two notions of atomicity to accommodate both kinds of weakest preconditions that we will define later:
If the weakest precondition ensures that the program cannot get stuck, weak atomicity is sufficient.
Otherwise, we need strong atomicity.

\begin{defn}[Context]
  A function $\lctx : \Expr \to \Expr$ is a \emph{context} if the following conditions are satisfied:
  \begin{enumerate}[itemsep=0pt]
  \item $\lctx$ does not turn non-values into values:\\
    $$\All\expr. \toval(\expr) = \bot \Ra \toval(\lctx(\expr)) = \bot $$
  \item One can perform reductions below $\lctx$:\\
    $$\All \expr_1, \state_1, \vec\obs, \expr_2, \state_2, \vec\expr. \expr_1, \state_1 \step[\vec\obs] \expr_2,\state_2,\vec\expr \Ra \lctx(\expr_1), \state_1 \step[\vec\obs] \lctx(\expr_2),\state_2,\vec\expr $$
  \item Reductions stay below $\lctx$ until there is a value in the hole:\\
    \begin{align*}
      &\All \expr_1', \state_1, \vec\obs, \expr_2, \state_2, \vec\expr. \toval(\expr_1') = \bot \land \lctx(\expr_1'), \state_1 \step[\vec\obs] \expr_2,\state_2,\vec\expr \Ra {}\\
      &\qquad \Exists\expr_2'. \expr_2 = \lctx(\expr_2') \land \expr_1', \state_1 \step[\vec\obs] \expr_2',\state_2,\vec\expr 
    \end{align*}
  \end{enumerate}
\end{defn}

\subsection{Concurrent Language}
\label{sec:language:concurrent}

For any language $\Lang$, we define the corresponding thread-pool semantics.

\paragraph{Machine syntax}
\[
	\tpool \in \ThreadPool \eqdef \List(\Expr)
\]

\judgment[Machine reduction]{\cfg{\tpool}{\state} \tpstep[\vec\obs]
  \cfg{\tpool'}{\state'}}
\begin{mathpar}
\infer
  {\expr_1, \state_1 \step[\vec\obs] \expr_2, \state_2, \vec\expr}
  {\cfg{\tpool \dplus [\expr_1] \dplus \tpool'}{\state_1} \tpstep[\vec\obs]
     \cfg{\tpool \dplus [\expr_2] \dplus \tpool' \dplus \vec\expr}{\state_2}}
\end{mathpar}

We use $\tpsteps[-]$ for the reflexive transitive closure of $\tpstep[-]$, as usual concatenating the lists of observations of the individual steps.


%%% Local Variables:
%%% mode: latex
%%% TeX-master: "iris"
%%% End:

\endgroup\clearpage\begingroup

\section{Program Logic}
\label{sec:program-logic}

This section describes how to build a program logic for an arbitrary language (\cf \Sref{sec:language}) on top of the base logic.
So in the following, we assume that some language $\Lang$ was fixed.

\subsection{Dynamic composeable higher-order resources}
\label{sec:composeable-resources}

The base logic described in \Sref{sec:base-logic} works over an arbitrary CMRA $\monoid$ defining the structure of the resources.
It turns out that we can generalize this further and permit picking CMRAs ``$\iFunc(\Prop)$'' that depend on the structure of assertions themselves.
Of course, $\Prop$ is just the syntactic type of assertions; for this to make sense we have to look at the semantics.

Furthermore, there is a composability problem with the given logic: if we have one proof performed with CMRA $\monoid_1$, and another proof carried out with a \emph{different} CMRA $\monoid_2$, then the two proofs are actually carried out in two \emph{entirely separate logics} and hence cannot be combined.

Finally, in many cases just having a single ``instance'' of a CMRA available for reasoning is not enough.
For example, when reasoning about a dynamically allocated data structure, every time a new instance of that data structure is created, we will want a fresh resource governing the state of this particular instance.
While it would be possible to handle this problem whenever it comes up, it turns out to be useful to provide a general solution.

The purpose of this section is to describe how we solve these issues.

\paragraph{Picking the resources.}
The key ingredient that we will employ on top of the base logic is to give some more fixed structure to the resources.
To instantiate the program logic, the user picks a family of locally contractive bifunctors $(\iFunc_i : \OFEs \to \CMRAs)_{i \in \mathcal{I}}$.
(This is in contrast to the base logic, where the user picks a single, fixed CMRA that has a unit.)

From this, we construct the bifunctor defining the overall resources as follows:
\begin{align*}
  \GName \eqdef{}& \nat \\
  \textdom{ResF}(\ofe^\op, \ofe) \eqdef{}& \prod_{i \in \mathcal I} \GName \fpfn \iFunc_i(\ofe^\op, \ofe)
\end{align*}
We will motivate both the use of a product and the finite partial function below.
$\textdom{ResF}(\ofe^\op, \ofe)$ is a CMRA by lifting the individual CMRAs pointwise, and it has a unit (using the empty finite partial functions).
Furthermore, since the $\iFunc_i$ are locally contractive, so is $\textdom{ResF}$.

Now we can write down the recursive domain equation:
\[ \iPreProp \cong \UPred(\textdom{ResF}(\iPreProp, \iPreProp)) \]
Here, $\iPreProp$ is a COFE defined as the fixed-point of a locally contractive bifunctor, which exists and is unique up to isomorphism by \thmref{thm:america_rutten}.
We do not need to consider how the object $\iPreProp$ is constructed, we only need the isomorphism, given by:
\begin{align*}
  \Res &\eqdef \textdom{ResF}(\iPreProp, \iPreProp) \\
  \iProp &\eqdef \UPred(\Res) \\
	\wIso &: \iProp \nfn \iPreProp \\
	\wIso^{-1} &: \iPreProp \nfn \iProp
\end{align*}

Notice that $\iProp$ is the semantic model of assertions for the base logic described in \Sref{sec:base-logic} with $\Res$:
\[ \Sem{\Prop} \eqdef \iProp = \UPred(\Res) \]
Effectively, we just defined a way to instantiate the base logic with $\Res$ as the CMRA of resources, while providing a way for $\Res$ to depend on $\iPreProp$, which is isomorphic to $\Sem\Prop$.

We thus obtain all the rules of \Sref{sec:base-logic}, and furthermore, we can use the maps $\wIso$ and $\wIso^{-1}$ \emph{in the logic} to convert between logical assertions $\Sem\Prop$ and the domain $\iPreProp$ which is used in the construction of $\Res$ -- so from elements of $\iPreProp$, we can construct elements of $\Sem{\textlog M}$, which are the elements that can be owned in our logic.

\paragraph{Proof composability.}
To make our proofs composeable, we \emph{generalize} our proofs over the family of functors.
This is possible because we made $\Res$ a \emph{product} of all the CMRAs picked by the user, and because we can actually work with that product ``pointwise''.
So instead of picking a \emph{concrete} family, proofs will assume to be given an \emph{arbitrary} family of functors, plus a proof that this family \emph{contains the functors they need}.
Composing two proofs is then merely a matter of conjoining the assumptions they make about the functors.
Since the logic is entirely parametric in the choice of functors, there is no trouble reasoning without full knowledge of the family of functors.

Only when the top-level proof is completed we will ``close'' the proof by picking a concrete family that contains exactly those functors the proof needs.

\paragraph{Dynamic resources.}
Finally, the use of finite partial functions lets us have as many instances of any CMRA as we could wish for:
Because there can only ever be finitely many instances already allocated, it is always possible to create a fresh instance with any desired (valid) starting state.
This is best demonstrated by giving some proof rules.

So let us first define the notion of ghost ownership that we use in this logic.
Assuming that the family of functors contains the functor $\Sigma_i$ at index $i$, and furthermore assuming that $\monoid_i = \Sigma_i(\iPreProp, \iPreProp)$, given some $\melt \in \monoid_i$ we define:
\[ \ownGhost\gname{\melt:\monoid_i} \eqdef \ownM{(\ldots, \emptyset, i:\mapsingleton \gname \melt, \emptyset, \ldots)} \]
This is ownership of the pair (element of the product over all the functors) that has the empty finite partial function in all components \emph{except for} the component corresponding to index $i$, where we own the element $\melt$ at index $\gname$ in the finite partial function.

We can show the following properties for this form of ownership:
\begin{mathparpagebreakable}
  \inferH{res-alloc}{\text{$G$ infinite} \and \melt \in \mval_{M_i}}
  {  \TRUE \proves \upd \Exists\gname\in G. \ownGhost\gname{\melt : M_i}
  }
  \and
  \inferH{res-update}
    {\melt \mupd_{M_i} B}
    {\ownGhost\gname{\melt : M_i} \proves \upd \Exists \meltB\in B. \ownGhost\gname{\meltB : M_i}}

  \inferH{res-empty}
  {\text{$\munit$ is a unit of $M_i$}}
  {\TRUE \proves \upd \ownGhost\gname\munit}
  
  \axiomH{res-op}
    {\ownGhost\gname{\melt : M_i} * \ownGhost\gname{\meltB : M_i} \provesIff \ownGhost\gname{\melt\mtimes\meltB : M_i}}

  \axiomH{res-valid}
    {\ownGhost\gname{\melt : M_i} \Ra \mval_{M_i}(\melt)}

  \inferH{res-timeless}
    {\text{$\melt$ is a discrete OFE element}}
    {\timeless{\ownGhost\gname{\melt : M_i}}}
\end{mathparpagebreakable}

Below, we will always work within (an instance of) the logic as described here.
Whenever a CMRA is used in a proof, we implicitly assume it to be available in the global family of functors.
We will typically leave the $M_i$ implicit when asserting ghost ownership, as the type of $\melt$ will be clear from the context.



\subsection{World satisfaction, invariants, fancy updates}
\label{sec:invariants}

To introduce invariants into our logic, we will define weakest precondition to explicitly thread through the proof that all the invariants are maintained throughout program execution.
However, in order to be able to access invariants, we will also have to provide a way to \emph{temporarily disable} (or ``open'') them.
To this end, we use tokens that manage which invariants are currently enabled.

We assume to have the following four CMRAs available:
\begin{align*}
  \InvName \eqdef{}& \nat \\
  \textmon{Inv} \eqdef{}& \authm(\InvName \fpfn \agm(\latert \iPreProp)) \\
  \textmon{En} \eqdef{}& \pset{\InvName} \\
  \textmon{Dis} \eqdef{}& \finpset{\InvName}
\end{align*}
The last two are the tokens used for managing invariants, $\textmon{Inv}$ is the monoid used to manage the invariants themselves.

We assume that at the beginning of the verification, instances named $\gname_{\textmon{State}}$, $\gname_{\textmon{Inv}}$, $\gname_{\textmon{En}}$ and $\gname_{\textmon{Dis}}$ of these CMRAs have been created, such that these names are globally known.

\paragraph{World Satisfaction.}
We can now define the assertion $W$ (\emph{world satisfaction}) which ensures that the enabled invariants are actually maintained:
\begin{align*}
  W \eqdef{}& \Exists I : \InvName \fpfn \Prop.
  \begin{array}[t]{@{} l}
    \ownGhost{\gname_{\textmon{Inv}}}{\authfull
      \mapComp {\iname}
        {\aginj(\latertinj(\wIso(I(\iname))))}
        {\iname \in \dom(I)}} * \\
    \Sep_{\iname \in \dom(I)} \left( \later I(\iname) * \ownGhost{\gname_{\textmon{Dis}}}{\set{\iname}} \lor \ownGhost{\gname_{\textmon{En}}}{\set{\iname}} \right)
  \end{array}
\end{align*}

\paragraph{Invariants.}
The following assertion states that an invariant with name $\iname$ exists and maintains assertion $\prop$:
\[ \knowInv\iname\prop \eqdef \ownGhost{\gname_{\textmon{Inv}}}
  {\authfrag \mapsingleton \iname {\aginj(\latertinj(\wIso(\prop)))}} \]

\paragraph{Fancy updates and view shifts.}
Next, we define \emph{fancy updates}, which are essentially the same as the basic updates of the base logic ($\Sref{sec:base-logic}$), except that they also have access to world satisfaction and can enable and disable invariants:
\[ \pvs[\mask_1][\mask_2] \prop \eqdef W * \ownGhost{\gname_{\textmon{En}}}{\mask_1} \wand \upd\diamond (W * \ownGhost{\gname_{\textmon{En}}}{\mask_2} * \prop) \]
Here, $\mask_1$ and $\mask_2$ are the \emph{masks} of the view update, defining which invariants have to be (at least!) available before and after the update.
We use $\top$ as symbol for the largest possible mask, $\nat$, and $\bot$ for the smallest possible mask $\emptyset$.
We will write $\pvs[\mask] \prop$ for $\pvs[\mask][\mask]\prop$.
%
Fancy updates satisfy the following basic proof rules:
\begin{mathparpagebreakable}
\infer[fup-mono]
{\prop \proves \propB}
{\pvs[\mask_1][\mask_2] \prop \proves \pvs[\mask_1][\mask_2] \propB}

\infer[fup-intro-mask]
{\mask_2 \subseteq \mask_1}
{\prop \proves \pvs[\mask_1][\mask_2]\pvs[\mask_2][\mask_1] \prop}

\infer[fup-trans]
{}
{\pvs[\mask_1][\mask_2] \pvs[\mask_2][\mask_3] \prop \proves \pvs[\mask_1][\mask_3] \prop}

\infer[fup-upd]
{}{\upd\prop \proves \pvs[\mask] \prop}

\infer[fup-frame]
{}{\propB * \pvs[\mask_1][\mask_2]\prop \proves \pvs[\mask_1 \uplus \mask_\f][\mask_2 \uplus \mask_\f] \propB * \prop}

\inferH{fup-update}
{\melt \mupd \meltsB}
{\ownM\melt \proves \pvs[\mask] \Exists\meltB\in\meltsB. \ownM\meltB}

\infer[fup-timeless]
{\timeless\prop}
{\later\prop \proves \pvs[\mask] \prop}
%
% \inferH{fup-allocI}
% {\text{$\mask$ is infinite}}
% {\later\prop \proves \pvs[\mask] \Exists \iname \in \mask. \knowInv\iname\prop}
%gov
% \inferH{fup-openI}
% {}{\knowInv\iname\prop \proves \pvs[\set\iname][\emptyset] \later\prop}
%
% \inferH{fup-closeI}
% {}{\knowInv\iname\prop \land \later\prop \proves \pvs[\emptyset][\set\iname] \TRUE}
\end{mathparpagebreakable}
(There are no rules related to invariants here. Those rules will be discussed later, in \Sref{sec:namespaces}.)

We can further define the notions of \emph{view shifts} and \emph{linear view shifts}:
\begin{align*}
  \prop \vsW[\mask_1][\mask_2] \propB \eqdef{}& \prop \wand \pvs[\mask_1][\mask_2] \propB \\
  \prop \vs[\mask_1][\mask_2] \propB \eqdef{}& \always(\prop \wand \pvs[\mask_1][\mask_2] \propB) \\
  \prop \vs[\mask] \propB \eqdef{}& \prop \vs[\mask][\mask] \propB
\end{align*}
These two are useful when writing down specifications and for comparing with previous versions of Iris, but for reasoning, it is typically easier to just work directly with fancy updates.
Still, just to give an idea of what view shifts ``are'', here are some proof rules for them:
\begin{mathparpagebreakable}
\inferH{vs-update}
  {\melt \mupd \meltsB}
  {\ownGhost\gname{\melt} \vs[\emptyset] \exists \meltB \in \meltsB.\; \ownGhost\gname{\meltB}}
\and
\inferH{vs-trans}
  {\prop \vs[\mask_1][\mask_2] \propB \and \propB \vs[\mask_2][\mask_3] \propC}
  {\prop \vs[\mask_1][\mask_3] \propC}
\and
\inferH{vs-imp}
  {\always{(\prop \Ra \propB)}}
  {\prop \vs[\emptyset] \propB}
\and
\inferH{vs-mask-frame}
  {\prop \vs[\mask_1][\mask_2] \propB}
  {\prop \vs[\mask_1 \uplus \mask'][\mask_2 \uplus \mask'] \propB}
\and
\inferH{vs-frame}
  {\prop \vs[\mask_1][\mask_2] \propB}
  {\prop * \propC \vs[\mask_1][\mask_2] \propB * \propC}
\and
\inferH{vs-timeless}
  {\timeless{\prop}}
  {\later \prop \vs[\emptyset] \prop}

% \inferH{vs-allocI}
%   {\infinite(\mask)}
%   {\later{\prop} \vs[\mask] \exists \iname\in\mask.\; \knowInv{\iname}{\prop}}
% \and
% \axiomH{vs-openI}
%   {\knowInv{\iname}{\prop} \proves \TRUE \vs[\{ \iname \} ][\emptyset] \later \prop}
% \and
% \axiomH{vs-closeI}
%   {\knowInv{\iname}{\prop} \proves \later \prop \vs[\emptyset][\{ \iname \} ] \TRUE }
%
\inferHB{vs-disj}
  {\prop \vs[\mask_1][\mask_2] \propC \and \propB \vs[\mask_1][\mask_2] \propC}
  {\prop \lor \propB \vs[\mask_1][\mask_2] \propC}
\and
\inferHB{vs-exist}
  {\All \var. (\prop \vs[\mask_1][\mask_2] \propB)}
  {(\Exists \var. \prop) \vs[\mask_1][\mask_2] \propB}
\and
\inferHB{vs-always}
  {\always\propB \proves \prop \vs[\mask_1][\mask_2] \propC}
  {\prop \land \always{\propB} \vs[\mask_1][\mask_2] \propC}
 \and
\inferH{vs-false}
  {}
  {\FALSE \vs[\mask_1][\mask_2] \prop }
\end{mathparpagebreakable}

\subsection{Weakest preconditions}

Finally, we can define the core piece of the program logic, the assertion that reasons about program behavior: Weakest precondition, from which Hoare triples will be derived.

\paragraph{Defining weakest precondition.}
We assume that everything making up the definition of the language, \ie values, expressions, states, the conversion functions, reduction relation and all their properties, are suitably reflected into the logic (\ie they are part of the signature $\Sig$).
We further assume (as a parameter) a predicate $\stateinterp : \State \to \iProp$ that interprets the physical state as an Iris assertion.
This can be instantiated, for example, with ownership of an authoritative RA to tie the physical state to fragments that are used for user-level proofs.

\begin{align*}
  \textdom{wp} \eqdef{}& \MU \textdom{wp}. \Lam \mask, \expr, \pred. \\
        & (\Exists\val. \toval(\expr) = \val \land \pvs[\mask] \pred(\val)) \lor {}\\
        & \Bigl(\toval(\expr) = \bot \land \All \state. \stateinterp(\state) \vsW[\mask][\emptyset] {}\\
        &\qquad \red(\expr, \state) * \later\All \expr', \state', \vec\expr. (\expr, \state \step \expr', \state', \vec\expr) \vsW[\emptyset][\mask] {}\\
        &\qquad\qquad \stateinterp(\state') * \textdom{wp}(\mask, \expr', \pred) * \Sep_{\expr'' \in \vec\expr} \textdom{wp}(\top, \expr'', \Lam \any. \TRUE)\Bigr) \\
%  (* value case *)
  \wpre\expr[\mask]{\Ret\val. \prop} \eqdef{}& \textdom{wp}(\mask, \expr, \Lam\val.\prop)
\end{align*}
If we leave away the mask, we assume it to default to $\top$.

\paragraph{Laws of weakest precondition.}
The following rules can all be derived:
\begin{mathpar}
\infer[wp-value]
{}{\prop[\val/\var] \proves \wpre{\val}[\mask]{\Ret\var.\prop}}

\infer[wp-mono]
{\mask_1 \subseteq \mask_2 \and \vctx,\var:\textlog{val}\mid\prop \proves \propB}
{\vctx\mid\wpre\expr[\mask_1]{\Ret\var.\prop} \proves \wpre\expr[\mask_2]{\Ret\var.\propB}}

\infer[fup-wp]
{}{\pvs[\mask] \wpre\expr[\mask]{\Ret\var.\prop} \proves \wpre\expr[\mask]{\Ret\var.\prop}}

\infer[wp-fup]
{}{\wpre\expr[\mask]{\Ret\var.\pvs[\mask] \prop} \proves \wpre\expr[\mask]{\Ret\var.\prop}}

\infer[wp-atomic]
{\physatomic{\expr}}
{\pvs[\mask_1][\mask_2] \wpre\expr[\mask_2]{\Ret\var. \pvs[\mask_2][\mask_1]\prop}
 \proves \wpre\expr[\mask_1]{\Ret\var.\prop}}

\infer[wp-frame]
{}{\propB * \wpre\expr[\mask]{\Ret\var.\prop} \proves \wpre\expr[\mask]{\Ret\var.\propB*\prop}}

\infer[wp-frame-step]
{\toval(\expr) = \bot \and \mask_2 \subseteq \mask_1}
{\wpre\expr[\mask_2]{\Ret\var.\prop} * \pvs[\mask_1][\mask_2]\later\pvs[\mask_2][\mask_1]\propB \proves \wpre\expr[\mask_1]{\Ret\var.\propB*\prop}}

\infer[wp-bind]
{\text{$\lctx$ is a context}}
{\wpre\expr[\mask]{\Ret\var. \wpre{\lctx(\ofval(\var))}[\mask]{\Ret\varB.\prop}} \proves \wpre{\lctx(\expr)}[\mask]{\Ret\varB.\prop}}
\end{mathpar}

We will also want a rule that connect weakest preconditions to the operational semantics of the language.
\begin{mathpar}
  \infer[wp-lift-step]
  {\toval(\expr_1) = \bot}
  { {\begin{inbox} % for some crazy reason, LaTeX is actually sensitive to the space between the "{ {" here and the "} }" below...
        ~~\All \state_1. \stateinterp(\state_1) \vsW[\mask][\emptyset] \red(\expr_1,\state_1) * {}\\\qquad~~ \later\All \expr_2, \state_2, \vec\expr.  (\expr_1, \state_1 \step \expr_2, \state_2, \vec\expr)  \vsW[\emptyset][\mask] \Bigl(\stateinterp(\state_2) * \wpre{\expr_2}[\mask]{\Ret\var.\prop} * \Sep_{\expr_\f \in \vec\expr} \wpre{\expr_\f}[\top]{\Ret\any.\TRUE}\Bigr)  {}\\\proves \wpre{\expr_1}[\mask]{\Ret\var.\prop}
      \end{inbox}} }
\end{mathpar}

% We can further derive some slightly simpler rules for special cases.
% \begin{mathparpagebreakable}
%   \infer[wp-lift-pure-step]
%   {\All \state_1. \red(\expr_1, \state_1) \and
%    \All \state_1, \expr_2, \state_2, \vec\expr. \expr_1,\state_1 \step \expr_2,\state_2,\vec\expr \Ra \state_1 = \state_2 }
%   {\later\All \state, \expr_2, \vec\expr. (\expr_1,\state \step \expr_2, \state,\vec\expr)  \Ra \wpre{\expr_2}[\mask]{\Ret\var.\prop} * \Sep_{\expr_\f \in \vec\expr} \wpre{\expr_\f}[\top]{\Ret\any.\TRUE} \proves \wpre{\expr_1}[\mask]{\Ret\var.\prop}}

%   \infer[wp-lift-atomic-step]
%   {\atomic(\expr_1) \and
%    \red(\expr_1, \state_1)}
%   { {\begin{inbox}~~\later\ownPhys{\state_1} * \later\All \val_2, \state_2, \vec\expr. (\expr_1,\state_1 \step \ofval(\val),\state_2,\vec\expr)  * \ownPhys{\state_2} \wand \prop[\val_2/\var] * \Sep_{\expr_\f \in \vec\expr} \wpre{\expr_\f}[\top]{\Ret\any.\TRUE} {}\\ \proves  \wpre{\expr_1}[\mask_1]{\Ret\var.\prop}
%   \end{inbox}} }

%   \infer[wp-lift-atomic-det-step]
%   {\atomic(\expr_1) \and
%    \red(\expr_1, \state_1) \and
%    \All \expr'_2, \state'_2, \vec\expr'. \expr_1,\state_1 \step \expr'_2,\state'_2,\vec\expr' \Ra \state_2 = \state_2' \land \toval(\expr_2') = \val_2 \land \vec\expr = \vec\expr'}
%   {\later\ownPhys{\state_1} * \later \Bigl(\ownPhys{\state_2} \wand \prop[\val_2/\var] * \Sep_{\expr_\f \in \vec\expr} \wpre{\expr_\f}[\top]{\Ret\any.\TRUE} \Bigr) \proves \wpre{\expr_1}[\mask_1]{\Ret\var.\prop}}

%   \infer[wp-lift-pure-det-step]
%   {\All \state_1. \red(\expr_1, \state_1) \\
%    \All \state_1, \expr_2', \state'_2, \vec\expr'. \expr_1,\state_1 \step \expr'_2,\state'_2,\vec\expr' \Ra \state_1 = \state'_2 \land \expr_2 = \expr_2' \land \vec\expr = \vec\expr'}
%   {\later \Bigl( \wpre{\expr_2}[\mask_1]{\Ret\var.\prop} * \Sep_{\expr_\f \in \vec\expr} \wpre{\expr_\f}[\top]{\Ret\any.\TRUE} \Bigr) \proves \wpre{\expr_1}[\mask_1]{\Ret\var.\prop}}
% \end{mathparpagebreakable}


\paragraph{Adequacy of weakest precondition.}

The purpose of the adequacy statement is to show that our notion of weakest preconditions is \emph{realistic} in the sense that it actually has anything to do with the actual behavior of the program.
There are two properties we are looking for: First of all, the postcondition should reflect actual properties of the values the program can terminate with.
Second, a proof of a weakest precondition with any postcondition should imply that the program is \emph{safe}, \ie that it does not get stuck.

\begin{defn}[Adequacy]
  A program $\expr$ in some initial state $\state$ is \emph{adequate} for a set $V \subseteq \Val$ of legal return values ($\expr, \state \vDash V$) if for all $\tpool', \state'$ such that $([\expr], \state) \tpstep^\ast (\tpool', \state')$ we have
\begin{enumerate}
\item Safety: For any $\expr' \in \tpool'$ we have that either $\expr'$ is a
  value, or \(\red(\expr'_i,\state')\):
  \[ \All\expr'\in\tpool'. \toval(\expr') \neq \bot \lor \red(\expr', \state') \]
  Notice that this is stronger than saying that the thread pool can reduce; we actually assert that \emph{every} non-finished thread can take a step.
\item Legal return value: If $\tpool'_1$ (the main thread) is a value $\val'$, then $\val' \in V$:
  \[ \All \val',\tpool''. \tpool' = [\val'] \dplus \tpool'' \Ra \val' \in V \]
\end{enumerate}
\end{defn}

To express the adequacy statement for functional correctness, we assume that the signature $\Sig$ adds a predicate $\pred$ to the logic:
\[ \pred : \Val \to \Prop \in \SigFn \]
Furthermore, we assume that the \emph{interpretation} $\Sem\pred$ of $\pred$ reflects some set $V$ of legal return values into the logic (also see \Sref{sec:model}):
\[\begin{array}{rMcMl}
  \Sem\pred &:& \Sem{\Val\,} \nfn \Sem\Prop \\
  \Sem\pred &\eqdef& \Lam \val. \Lam \any. \setComp{n}{v \in V}
\end{array}\]
The signature can of course state arbitrary additional properties of $\pred$, as long as they are proven sound.
The adequacy statement now reads as follows:
\begin{align*}
 &\All \mask, \expr, \val, \state.
 \\&( \TRUE \proves {\upd}_\mask \Exists \stateinterp. \stateinterp(\state) * \wpre{\expr}[\mask]{x.\; \pred(x)}) \Ra
 \\&\expr, \state \vDash V
\end{align*}
Notice that the state invariant $S$ used by the weakest precondition is chosen \emph{after} doing a fancy update, which allows it to depend on the names of ghost variables that are picked in that initial fancy update.

\paragraph{Hoare triples.}
It turns out that weakest precondition is actually quite convenient to work with, in particular when performing these proofs in Coq.
Still, for a more traditional presentation, we can easily derive the notion of a Hoare triple:
\[
\hoare{\prop}{\expr}{\Ret\val.\propB}[\mask] \eqdef \always{(\prop \wand \wpre{\expr}[\mask]{\Ret\val.\propB})}
\]

We only give some of the proof rules for Hoare triples here, since we usually do all our reasoning directly with weakest preconditions and use Hoare triples only to write specifications.
\begin{mathparpagebreakable}
\inferH{Ht-ret}
  {}
  {\hoare{\TRUE}{\valB}{\Ret\val. \val = \valB}[\mask]}
\and
\inferH{Ht-bind}
  {\text{$\lctx$ is a context} \and \hoare{\prop}{\expr}{\Ret\val. \propB}[\mask] \\
   \All \val. \hoare{\propB}{\lctx(\val)}{\Ret\valB.\propC}[\mask]}
  {\hoare{\prop}{\lctx(\expr)}{\Ret\valB.\propC}[\mask]}
\and
\inferH{Ht-csq}
  {\prop \vs \prop' \\
    \hoare{\prop'}{\expr}{\Ret\val.\propB'}[\mask] \\   
   \All \val. \propB' \vs \propB}
  {\hoare{\prop}{\expr}{\Ret\val.\propB}[\mask]}
\and
% \inferH{Ht-mask-weaken}
%   {\hoare{\prop}{\expr}{\Ret\val. \propB}[\mask]}
%   {\hoare{\prop}{\expr}{\Ret\val. \propB}[\mask \uplus \mask']}
% \\\\
\inferH{Ht-frame}
  {\hoare{\prop}{\expr}{\Ret\val. \propB}[\mask]}
  {\hoare{\prop * \propC}{\expr}{\Ret\val. \propB * \propC}[\mask]}
\and
% \inferH{Ht-frame-step}
%   {\hoare{\prop}{\expr}{\Ret\val. \propB}[\mask] \and \toval(\expr) = \bot \and \mask_2 \subseteq \mask_2 \\\\ \propC_1 \vs[\mask_1][\mask_2] \later\propC_2 \and \propC_2 \vs[\mask_2][\mask_1] \propC_3}
%   {\hoare{\prop * \propC_1}{\expr}{\Ret\val. \propB * \propC_3}[\mask \uplus \mask_1]}
% \and
\inferH{Ht-atomic}
  {\prop \vs[\mask \uplus \mask'][\mask] \prop' \\
    \hoare{\prop'}{\expr}{\Ret\val.\propB'}[\mask] \\   
   \All\val. \propB' \vs[\mask][\mask \uplus \mask'] \propB \\
   \physatomic{\expr}
  }
  {\hoare{\prop}{\expr}{\Ret\val.\propB}[\mask \uplus \mask']}
\and
\inferH{Ht-false}
  {}
  {\hoare{\FALSE}{\expr}{\Ret \val. \prop}[\mask]}
\and
\inferHB{Ht-disj}
  {\hoare{\prop}{\expr}{\Ret\val.\propC}[\mask] \and \hoare{\propB}{\expr}{\Ret\val.\propC}[\mask]}
  {\hoare{\prop \lor \propB}{\expr}{\Ret\val.\propC}[\mask]}
\and
\inferHB{Ht-exist}
  {\All \var. \hoare{\prop}{\expr}{\Ret\val.\propB}[\mask]}
  {\hoare{\Exists \var. \prop}{\expr}{\Ret\val.\propB}[\mask]}
\and
\inferHB{Ht-box}
  {\always\propB \proves \hoare{\prop}{\expr}{\Ret\val.\propC}[\mask]}
  {\hoare{\prop \land \always{\propB}}{\expr}{\Ret\val.\propC}[\mask]}
% \and
% \inferH{Ht-inv}
%   {\hoare{\later\propC*\prop}{\expr}{\Ret\val.\later\propC*\propB}[\mask] \and
%    \physatomic{\expr}
%   }
%   {\knowInv\iname\propC \proves \hoare{\prop}{\expr}{\Ret\val.\propB}[\mask \uplus \set\iname]}
% \and
% \inferH{Ht-inv-timeless}
%   {\hoare{\propC*\prop}{\expr}{\Ret\val.\propC*\propB}[\mask] \and
%    \physatomic{\expr} \and \timeless\propC
%   }
%   {\knowInv\iname\propC \proves \hoare{\prop}{\expr}{\Ret\val.\propB}[\mask \uplus \set\iname]}
\end{mathparpagebreakable}

\subsection{Invariant namespaces}
\label{sec:namespaces}

In \Sref{sec:invariants}, we defined an assertion $\knowInv\iname\prop$ expressing knowledge (\ie the assertion is persistent) that $\prop$ is maintained as invariant with name $\iname$.
The concrete name $\iname$ is picked when the invariant is allocated, so it cannot possibly be statically known -- it will always be a variable that's threaded through everything.
However, we hardly care about the actual, concrete name.
All we need to know is that this name is \emph{different} from the names of other invariants that we want to open at the same time.
Keeping track of the $n^2$ mutual inequalities that arise with $n$ invariants quickly gets in the way of the actual proof.

To solve this issue, instead of remembering the exact name picked for an invariant, we will keep track of the \emph{namespace} the invariant was allocated in.
Namespaces are sets of invariants, following a tree-like structure:
Think of the name of an invariant as a sequence of identifiers, much like a fully qualified Java class name.
A \emph{namespace} $\namesp$ then is like a Java package: it is a sequence of identifiers that we think of as \emph{containing} all invariant names that begin with this sequence. For example, \texttt{org.mpi-sws.iris} is a namespace containing the invariant name \texttt{org.mpi-sws.iris.heap}.

The crux is that all namespaces contain infinitely many invariants, and hence we can \emph{freely pick} the namespace an invariant is allocated in -- no further, unpredictable choice has to be made.
Furthermore, we will often know that namespaces are \emph{disjoint} just by looking at them.
The namespaces $\namesp.\texttt{iris}$ and $\namesp.\texttt{gps}$ are disjoint no matter the choice of $\namesp$.
As a result, there is often no need to track disjointness of namespaces, we just have to pick the namespaces that we allocate our invariants in accordingly.

Formally speaking, let $\namesp \in \textlog{InvNamesp} \eqdef \List(\nat)$ be the type of \emph{invariant namespaces}.
We use the notation $\namesp.\iname$ for the namespace $[\iname] \dplus \namesp$.
(In other words, the list is ``backwards''. This is because cons-ing to the list, like the dot does above, is easier to deal with in Coq than appending at the end.)

The elements of a namespaces are \emph{structured invariant names} (think: Java fully qualified class name).
They, too, are lists of $\nat$, the same type as namespaces.
In order to connect this up to the definitions of \Sref{sec:invariants}, we need a way to map structued invariant names to $\InvName$, the type of ``plain'' invariant names.
Any injective mapping $\textlog{namesp\_inj}$ will do; and such a mapping has to exist because $\List(\nat)$ is countable and $\InvName$ is infinite.
Whenever needed, we (usually implicitly) coerce $\namesp$ to its encoded suffix-closure, \ie to the set of encoded structured invariant names contained in the namespace: \[\namecl\namesp \eqdef \setComp{\iname}{\Exists \namesp'. \iname = \textlog{namesp\_inj}(\namesp' \dplus \namesp)}\]

We will overload the notation for invariant assertions for using namespaces instead of names:
\[ \knowInv\namesp\prop \eqdef \Exists \iname \in \namecl\namesp. \knowInv\iname{\prop} \]
We can now derive the following rules (this involves unfolding the definition of fancy updates):
\begin{mathpar}
  \axiomH{inv-persist}{\knowInv\namesp\prop \proves \always\knowInv\namesp\prop}

  \axiomH{inv-alloc}{\later\prop \proves \pvs[\emptyset] \knowInv\namesp\prop}

  \inferH{inv-open}
  {\namesp \subseteq \mask}
  {\knowInv\namesp\prop \vs[\mask][\mask\setminus\namesp] \later\prop * (\later\prop \vsW[\mask\setminus\namesp][\mask] \TRUE)}

  \inferH{inv-open-timeless}
  {\namesp \subseteq \mask \and \timeless\prop}
  {\knowInv\namesp\prop \vs[\mask][\mask\setminus\namesp] \prop * (\prop \vsW[\mask\setminus\namesp][\mask] \TRUE)}
\end{mathpar}

\subsection{Accessors}

The two rules \ruleref{inv-open} and \ruleref{inv-open-timeless} above may look a little surprising, in the sense that it is not clear on first sight how they would be applied.
The rules are the first \emph{accessors} that show up in this document.
Accessors are assertions of the form
\[ \prop \vs[\mask_1][\mask_2] \Exists\var. \propB * (\All\varB. \propB' \vsW[\mask_2][\mask_1] \propC) \]

One way to think about such assertions is as follows:
Given some accessor, if during our verification we have the assertion $\prop$ and the mask $\mask_1$ available, we can use the accessor to \emph{access} $\propB$ and obtain the witness $\var$.
We call this \emph{opening} the accessor, and it changes the mask to $\mask_2$.
Additionally, opening the accessor provides us with $\All\varB. \propB' \vsW[\mask_2][\mask_1] \propC$, a \emph{linear view shift} (\ie a view shift that can only be used once).
This linear view shift tells us that in order to \emph{close} the accessor again and go back to mask $\mask_1$, we have to pick some $\varB$ and establish the corresponding $\propB'$.
After closing, we will obtain $\propC$.

Using \ruleref{vs-trans} and \ruleref{Ht-atomic} (or the corresponding proof rules for fancy updates and weakest preconditions), we can show that it is possible to open an accessor around any view shift and any \emph{atomic} expression:
\begin{mathpar}
  \inferH{Acc-vs}
  {\prop \vs[\mask_1][\mask_2] \Exists\var. \propB * (\All\varB. \propB' \vsW[\mask_2][\mask_1] \propC) \and
   \All\var. \propB * \prop_F \vs[\mask_2] \Exists\varB. \propB' * \prop_F}
  {\prop * \prop_F \vs[\mask_1] \propC * \prop_F}

  \inferH{Acc-Ht}
  {\prop \vs[\mask_1][\mask_2] \Exists\var. \propB * (\All\varB. \propB' \vsW[\mask_2][\mask_1] \propC) \and
   \All\var. \hoare{\propB * \prop_F}\expr{\Exists\varB. \propB' * \prop_F}[\mask_2] \and
   \physatomic\expr}
  {\hoare{\prop * \prop_F}\expr{\propC * \prop_F}[\mask_1]}
\end{mathpar}

Furthermore, in the special case that $\mask_1 = \mask_2$, the accessor can be opened around \emph{any} expression.
For this reason, we also call such accessors \emph{non-atomic}.

The reasons accessors are useful is that they let us talk about ``opening X'' (\eg ``opening invariants'') without having to care what X is opened around.
Furthermore, as we construct more sophisticated and more interesting things that can be opened (\eg invariants that can be ``cancelled'', or STSs), accessors become a useful interface that allows us to mix and match different abstractions in arbitrary ways.

For the special case that $\prop = \propC$ and $\propB = \propB'$, we use the following notation that avoids repetition:
\[ \Acc[\mask_1][\mask_2]\prop{\Ret x. \propB} \eqdef \prop \vs[\mask_1][\mask_2] \Exists\var. \propB * (\propB \vsW[\mask_2][\mask_1] \prop)  \]
This accessor is ``idempotent'' in the sense that it doesn't actually change the state.  After applying it, we get our $\prop$ back so we end up where we started.

%%% Local Variables:
%%% mode: latex
%%% TeX-master: "iris"
%%% End:

\endgroup\clearpage\begingroup
\section{Derived Constructions}

\subsection{Cancellable Invariants}

Iris invariants as described in \Sref{sec:invariants} are persistent---once established, they hold forever.
However, based on them, it is possible to \emph{encode} a form of invariants that can be ``cancelled'' again.

First, we need some ghost state:
\begin{align*}
  \textdom{CInvTok} \eqdef{}& \fracm
\end{align*}

Now we define:
\begin{align*}
  \CInvTok{\gname}{q} \eqdef{}& \ownGhost\gname{q} \\
  \CInv{\gname}{\namesp}{\prop} \eqdef{}& \knowInv\namesp{\prop \lor \ownGhost\gname{1}}
\end{align*}

It is then straightforward to prove:
\begin{mathpar}
  \inferH{CInv-new}{}
  {\later\prop \vs[\bot] \Exists \gname. \CInvTok\gname{1} * \always\CInv\gname\namesp\prop}

  \inferH{CInv-acc}{}
  {\CInv\gname\namesp\prop \proves \Acc[\namesp][\emptyset]{\CInvTok\gname{q}}{\later\prop}}

  \inferH{CInv-cancel}{}
  {\CInv\gname\namesp\prop \proves \CInvTok\gname{1} \vs[\namesp] \later\prop}
\end{mathpar}

Cancellable invariants are useful, for example, when reasoning about data structures that will be deallocated:  Every reference to the data structure comes with a fraction of the token, and when all fractions have been gathered, \ruleref{CInv-cancel} is used to cancel the invariant, after which the data structure can be deallocated.

\subsection{Non-atomic (``Thread-Local'') Invariants}

Sometimes it is necessary to maintain invariants that we need to open non-atomically.
Clearly, for this mechanism to be sound we need something that prevents us from opening the same invariant twice, something like the masks that avoid reentrancy on the ``normal'', atomic invariants.
The idea is to use tokens\footnote{Very much like the tokens that are used to encode ``normal'', atomic invariants} that guard access to non-atomic invariants.
Having the token $\NaTokE\pid\mask$ indicates that we can open all invariants in $\mask$.
The $\pid$ here is the name of the \emph{invariant pool}.
This mechanism allows us to have multiple, independent pools of invariants that all have their own namespaces.

One way to think about non-atomic invariants is as ``thread-local invariants'',
where every pool is a thread.
Every thread thus has its own, independent set of invariants.
Every thread threads through all the tokens for its own pool, so that each invariant can only be opened in the thread it belongs to.
As a consequence, they can be kept open around any sequence of expressions (\ie there is no restriction to atomic expressions) -- after all, there cannot be any races with other threads.

Concretely, this is the monoid structure we need:
\begin{align*}
\textdom{PId} \eqdef{}& \GName \\
\textdom{NaTok} \eqdef{}& \finpset{\InvName} \times \pset{\InvName}
\end{align*}
For every pool, there is a set of tokens designating which invariants are \emph{enabled} (closed).
This corresponds to the mask of ``normal'' invariants.
We re-use the structure given by namespaces for non-atomic invariants.
Furthermore, there is a \emph{finite} set of invariants that is \emph{disabled} (open).

Owning tokens is defined as follows:
\begin{align*}
\NaTokE\pid\mask \eqdef{}& \ownGhost{\pid}{ (\emptyset, \mask) } \\
\NaTok\pid \eqdef{}& \NaTokE\pid\top
\end{align*}

Next, we define non-atomic invariants.
To simplify this construction,we piggy-back into ``normal'' invariants.
\begin{align*}
  \NaInv\pid\namesp\prop \eqdef{}& \Exists \iname\in\namesp. \knowInv\namesp{\prop * \ownGhost\pid{(\set{\iname},\emptyset)} \lor \NaTokE\pid{\set{\iname}}}
\end{align*}


We easily obtain:
\begin{mathpar}
  \axiomH{NAInv-new-pool}
  {\TRUE \vs[\bot] \Exists\pid. \NaTok\pid}

  \axiomH{NAInv-tok-split}
  {\NaTokE\pid{\mask_1 \uplus \mask_2} \Lra \NaTokE\pid{\mask_1} * \NaTokE\pid{\mask_2}}
  
  \axiomH{NAInv-new-inv}
  {\later\prop  \vs[\namesp] \always\NaInv\pid\namesp\prop}

  \axiomH{NAInv-acc}
  {\NaInv\pid\namesp\prop \proves \Acc[\namesp]{\NaTokE\pid\namesp}{\later\prop}}
\end{mathpar}
from which we can derive
\begin{mathpar}
  \infer
  {\namesp \subseteq \mask}
  {\NaInv\pid\namesp\prop \proves \Acc[\namesp]{\NaTokE\pid\mask}{\later\prop * \NaTokE\pid{\mask \setminus \namesp}}}
\end{mathpar}

\subsection{Boxes}

The idea behind the \emph{boxes} is to have an proposition $\prop$ that is actually split into a number of pieces, each of which can be taken out and back in separately.
In some sense, this is a replacement for having an ``authoritative PCM of Iris propositions itself''.
It is similar to the pattern involving saved propositions that was used for the barrier~\cite{iris2}, but more complicated because there are some operations that we want to perform without a later.

Roughly, the idea is that a \emph{box} is a container for an proposition $\prop$.
A box consists of a bunch of \emph{slices} which decompose $\prop$ into a separating conjunction of the propositions $\propB_\sname$ governed by the individual slices.
Each slice is either \emph{full} (it right now contains $\propB_\sname$), or \emph{empty} (it does not contain anything currently).
The proposition governing the box keeps track of the state of all the slices that make up the box.
The crux is that opening and closing of a slice can be done even if we only have ownership of the boxes ``later'' ($\later$).

The interface for boxes is as follows:
The two core propositions are: $\BoxSlice\namesp\prop\sname$, saying that there is a slice in namespace $\namesp$ with name $\sname$  and content $\prop$; and $\ABox\namesp\prop{f}$, saying that $f$ describes the slices of a box in namespace $\namesp$, such that all the slices together contain $\prop$. Here, $f$ is of type $\nat \fpfn \BoxState$ mapping names to states, where $\BoxState \eqdef \set{\BoxFull, \BoxEmp}$.
\begin{mathpar}
  \inferH{Box-create}{}
  {\TRUE \vs[\namesp] \ABox\namesp\TRUE\emptyset}

  \inferH{Slice-insert-empty}{}
  {\lateropt b\ABox\namesp\prop{f} \vs[\namesp] \Exists\sname \notin \dom(f). \always\BoxSlice\namesp\propB\sname * \lateropt b\ABox\namesp{\prop * \propB}{\mapinsert\sname\BoxEmp{f}}}

  \inferH{Slice-delete-empty}
  {f(\sname) = \BoxEmp}
  {\BoxSlice\namesp\propB\sname \proves \lateropt b\ABox\namesp\prop{f} \vs[\namesp] \Exists \prop'. \lateropt b(\later(\prop = \prop' * \propB) * \ABox\namesp{\prop'}{\mapinsert\sname\bot{f}})}

  \inferH{Slice-fill}
  {f(\sname) = \BoxEmp}
  {\BoxSlice\namesp\propB\sname \proves \lateropt b\propB * \later\ABox\namesp\prop{f} \vs[\namesp] \lateropt b\ABox\namesp\prop{\mapinsert\sname\BoxFull{f}}}

  \inferH{Slice-empty}
  {f(\sname) = \BoxFull}
  {\BoxSlice\namesp\propB\sname \proves \lateropt b\ABox\namesp\prop{f} \vs[\namesp] \later\propB * \lateropt b\ABox\namesp\prop{\mapinsert\sname\BoxEmp{f}}}

  \inferH{Box-fill}
  {\All\sname\in\dom(f). f(\sname) = \BoxEmp}
  {\later\prop * \ABox\namesp\prop{f} \vs[\namesp] \ABox\namesp\prop{\mapinsertComp\sname\BoxFull{\sname\in\dom(f)}{f}}}

  \inferH{Box-empty}
  {\All\sname\in\dom(f). f(\sname) = \BoxFull}
  {\ABox\namesp\prop{f} \vs[\namesp] \later\prop * \ABox\namesp\prop{\mapinsertComp\sname\BoxEmp{\sname\in\dom(f)}{f}}}
\end{mathpar}
Above, $\lateropt b \prop$ is syntactic sugar for $\later\prop$ (if $b$ is $1$) or $\prop$ (if $b$ is $0$).
This is essentially an \emph{optional later}, indicating that the lemmas can be applied with \textlog{Box} being owned now or later, and that ownership is returned the same way.

\begingroup
\paragraph{Model.}
\newcommand\BoxM{\textdom{Box}}
\newcommand\SliceInv{\textlog{SliceInv}}

The above rules are validated by the following model.
We need a camera as follows:
\begin{align*}
  \BoxState \eqdef{}& \BoxFull + \BoxEmp \\
  \BoxM \eqdef{}& \authm(\maybe{\exm(\BoxState)}) \times \maybe{\agm(\latert \iProp)}
\end{align*}

Now we can define the propositions:
\begin{align*}
  \SliceInv(\sname, \prop) \eqdef{}& \Exists b. \ownGhost\sname{(\authfull b, \munit)} * ((b = \BoxFull) \Ra \prop) \\
  \BoxSlice\namesp\prop\sname \eqdef{}& \ownGhost\sname{(\munit, \prop)} * \knowInv\namesp{\SliceInv(\sname,\prop)} \\
  \ABox\namesp\prop{f} \eqdef{}& \Exists \propB : \nat \to \Prop. \later\left( \prop = \Sep_{\sname \in \dom(f)} \propB(\sname) \right ) * {}\\
  & \Sep_{\sname \in \dom(f)} \ownGhost\sname{(\authfrag f(\sname), \propB(\sname))} * \knowInv\namesp{\SliceInv(\sname,\propB(\sname))}
\end{align*}
\endgroup % Model paragraph

\paragraph{Derived rules.}
Here are some derived rules:
\begin{mathpar}
  \inferH{Slice-insert-full}{}
  {\later\propB * \lateropt b\ABox\namesp\prop{f} \vs[\namesp] \Exists\sname \notin \dom(f). \always\BoxSlice\namesp\propB\sname * \lateropt b\ABox\namesp{\prop * \propB}{\mapinsert\sname\BoxFull{f}}}

  \inferH{Slice-delete-full}
  {f(\sname) = \BoxFull}
  {\BoxSlice\namesp\propB\sname \proves \lateropt b \ABox\namesp\prop{f} \vs[\namesp] \later\propB * \Exists \prop'. \lateropt b (\later(\prop = \prop' * \propB) * \ABox\namesp{\prop'}{\mapinsert\sname\bot{f}})}

  \inferH{Slice-split}
  {f(\sname) = s}
  {\kern-4ex\BoxSlice\namesp{\propB_1 * \propB_2}\sname \proves \lateropt b \ABox\namesp\prop{f} \vs[\namesp] \Exists \sname_1 \notin \dom(f), \sname_2 \notin \dom(f). \sname_1 \neq \sname_2 \land {}\\\kern5ex \always\BoxSlice\namesp{\propB_1}{\sname_1} * \always\BoxSlice\namesp{\propB_2}{\sname_2} * \lateropt b \ABox\namesp\prop{\mapinsert{\sname_2}{s}{\mapinsert{\sname_1}{s}{\mapinsert\sname\bot{f}}}}}

  \inferH{Slice-merge}
  {\sname_1 \neq \sname_2 \and f(\sname_1) = f(\sname_2) = s}
  {\BoxSlice\namesp{\propB_1}{\sname_1}, \BoxSlice\namesp{\propB_2}{\sname_2} \proves \lateropt b \ABox\namesp\prop{f} \vs[\namesp] \Exists \sname \notin \dom(f) \setminus \set{\sname_1, \sname_2}. {}\\\kern5ex \always\BoxSlice\namesp{\propB_1 * \propB_2}\sname * \lateropt b \ABox\namesp\prop{\mapinsert\sname{s}{\mapinsert{\sname_2}{\bot}{\mapinsert{\sname_1}{\bot}{f}}}}}
\end{mathpar}




% TODO: These need syncing with Coq
% \subsection{STSs with interpretation}\label{sec:stsinterp}

% Building on \Sref{sec:stsmon}, after constructing the monoid $\STSMon{\STSS}$ for a particular STS, we can use an invariant to tie an interpretation, $\pred : \STSS \to \Prop$, to the STS's current state, recovering CaReSL-style reasoning~\cite{caresl}.

% An STS invariant asserts authoritative ownership of an STS's current state and that state's interpretation:
% \begin{align*}
%   \STSInv(\STSS, \pred, \gname) \eqdef{}& \Exists s \in \STSS. \ownGhost{\gname}{(s, \STSS, \emptyset):\STSMon{\STSS}} * \pred(s) \\
%   \STS(\STSS, \pred, \gname, \iname) \eqdef{}& \knowInv{\iname}{\STSInv(\STSS, \pred, \gname)}
% \end{align*}

% We can specialize \ruleref{NewInv}, \ruleref{InvOpen}, and \ruleref{InvClose} to STS invariants:
% \begin{mathpar}
%  \inferH{NewSts}
%   {\infinite(\mask)}
%   {\later\pred(s) \vs[\mask] \Exists \iname \in \mask, \gname.   \STS(\STSS, \pred, \gname, \iname) * \ownGhost{\gname}{(s, \STST \setminus \STSL(s)) : \STSMon{\STSS}}}
%  \and
%  \axiomH{StsOpen}
%   {  \STS(\STSS, \pred, \gname, \iname) \vdash \ownGhost{\gname}{(s_0, T) : \STSMon{\STSS}} \vsE[\{\iname\}][\emptyset] \Exists s\in \upclose(\{s_0\}, T). \later\pred(s) * \ownGhost{\gname}{(s, \upclose(\{s_0\}, T), T):\STSMon{\STSS}}}
%  \and
%  \axiomH{StsClose}
%   {  \STS(\STSS, \pred, \gname, \iname), (s, T) \ststrans (s', T')  \proves \later\pred(s') * \ownGhost{\gname}{(s, S, T):\STSMon{\STSS}} \vs[\emptyset][\{\iname\}] \ownGhost{\gname}{(s', T') : \STSMon{\STSS}} }
% \end{mathpar}
% \begin{proof}
% \ruleref{NewSts} uses \ruleref{NewGhost} to allocate $\ownGhost{\gname}{(s, \upclose(s, T), T) : \STSMon{\STSS}}$ where $T \eqdef \STST \setminus \STSL(s)$, and \ruleref{NewInv}.

% \ruleref{StsOpen} just uses \ruleref{InvOpen} and \ruleref{InvClose} on $\iname$, and the monoid equality $(s, \upclose(\{s_0\}, T), T) = (s, \STSS, \emptyset) \mtimes (\munit, \upclose(\{s_0\}, T), T)$.

% \ruleref{StsClose} applies \ruleref{StsStep} and \ruleref{InvClose}.
% \end{proof}

% Using these view shifts, we can prove STS variants of the invariant rules \ruleref{Inv} and \ruleref{VSInv}~(compare the former to CaReSL's island update rule~\cite{caresl}):
% \begin{mathpar}
%  \inferH{Sts}
%   {\All s \in \upclose(\{s_0\}, T). \hoare{\later\pred(s) * P}{\expr}{\Ret \val. \Exists s', T'. (s, T) \ststrans (s', T') * \later\pred(s') * Q}[\mask]
%    \and \physatomic{\expr}}
%   {  \STS(\STSS, \pred, \gname, \iname) \vdash \hoare{\ownGhost{\gname}{(s_0, T):\STSMon{\STSS}} * P}{\expr}{\Ret \val. \Exists s', T'. \ownGhost{\gname}{(s', T'):\STSMon{\STSS}} * Q}[\mask \uplus \{\iname\}]}
%  \and
%  \inferH{VSSts}
%   {\forall s \in \upclose(\{s_0\}, T).\; \later\pred(s) * P \vs[\mask_1][\mask_2] \exists s', T'.\; (s, T) \ststrans (s', T') * \later\pred(s') * Q}
%   {  \STS(\STSS, \pred, \gname, \iname) \vdash \ownGhost{\gname}{(s_0, T):\STSMon{\STSS}} * P \vs[\mask_1 \uplus \{\iname\}][\mask_2 \uplus \{\iname\}] \Exists s', T'. \ownGhost{\gname}{(s', T'):\STSMon{\STSS}} * Q}
% \end{mathpar}

% \begin{proof}[Proof of \ruleref{Sts}]\label{pf:sts}
%  We have to show
%  \[\hoare{\ownGhost{\gname}{(s_0, T):\STSMon{\STSS}} * P}{\expr}{\Ret \val. \Exists s', T'. \ownGhost{\gname}{(s', T'):\STSMon{\STSS}} * Q}[\mask \uplus \{\iname\}]\]
%  where $\val$, $s'$, $T'$ are free in $Q$.
 
%  First, by \ruleref{ACsq} with \ruleref{StsOpen} and \ruleref{StsClose} (after moving $(s, T) \ststrans (s', T')$ into the view shift using \ruleref{VSBoxOut}), it suffices to show
%  \[\hoareV{\Exists s\in \upclose(\{s_0\}, T). \later\pred(s) * \ownGhost{\gname}{(s, \upclose(\{s_0\}, T), T)} * P}{\expr}{\Ret \val. \Exists s, T, S, s', T'. (s, T) \ststrans (s', T') * \later\pred(s') * \ownGhost{\gname}{(s, S, T):\STSMon{\STSS}} * Q(\val, s', T')}[\mask]\]

%  Now, use \ruleref{Exist} to move the $s$ from the precondition into the context and use \ruleref{Csq} to (i)~fix the $s$ and $T$ in the postcondition to be the same as in the precondition, and (ii)~fix $S \eqdef \upclose(\{s_0\}, T)$.
%  It remains to show:
%  \[\hoareV{s\in \upclose(\{s_0\}, T) * \later\pred(s) * \ownGhost{\gname}{(s, \upclose(\{s_0\}, T), T)} * P}{\expr}{\Ret \val. \Exists s', T'. (s, T) \ststrans (s', T') * \later\pred(s') * \ownGhost{\gname}{(s, \upclose(\{s_0\}, T), T)} * Q(\val, s', T')}[\mask]\]
 
%  Finally, use \ruleref{BoxOut} to move $s\in \upclose(\{s_0\}, T)$ into the context, and \ruleref{Frame} on $\ownGhost{\gname}{(s, \upclose(\{s_0\}, T), T)}$:
%  \[s\in \upclose(\{s_0\}, T) \vdash \hoare{\later\pred(s) * P}{\expr}{\Ret \val. \Exists s', T'. (s, T) \ststrans (s', T') * \later\pred(s') * Q(\val, s', T')}[\mask]\]
 
%  This holds by our premise.
% \end{proof}

% % \begin{proof}[Proof of \ruleref{VSSts}]
% % This is similar to above, so we only give the proof in short notation:

% % \hproof{%
% % 	Context: $\knowInv\iname{\STSInv(\STSS, \pred, \gname)}$ \\
% % 	\pline[\mask_1 \uplus \{\iname\}]{
% % 		\ownGhost\gname{(s_0, T)} * P
% % 	} \\
% % 	\pline[\mask_1]{%
% % 		\Exists s. \later\pred(s) * \ownGhost\gname{(s, S, T)} * P
% % 	} \qquad by \ruleref{StsOpen} \\
% % 	Context: $s \in S \eqdef \upclose(\{s_0\}, T)$ \\
% % 	\pline[\mask_2]{%
% % 		 \Exists s', T'. \later\pred(s') * Q(s', T') * \ownGhost\gname{(s, S, T)}
% % 	} \qquad by premiss \\
% % 	Context: $(s, T) \ststrans (s', T')$ \\
% % 	\pline[\mask_2 \uplus \{\iname\}]{
% % 		\ownGhost\gname{(s', T')} * Q(s', T')
% % 	} \qquad by \ruleref{StsClose}
% % }
% % \end{proof}

% \subsection{Authoritative monoids with interpretation}\label{sec:authinterp}

% Building on \Sref{sec:auth}, after constructing the monoid $\auth{M}$ for a cancellative monoid $M$, we can tie an interpretation, $\pred : \mcarp{M} \to \Prop$, to the authoritative element of $M$, recovering reasoning that is close to the sharing rule in~\cite{krishnaswami+:icfp12}.

% Let $\pred_\bot$ be the extension of $\pred$ to $\mcar{M}$ with $\pred_\bot(\mzero) = \FALSE$.
% Now define
% \begin{align*}
%   \AuthInv(M, \pred, \gname) \eqdef{}& \exists \melt \in \mcar{M}.\; \ownGhost{\gname}{\authfull \melt:\auth{M}} * \pred_\bot(\melt) \\
%   \Auth(M, \pred, \gname, \iname) \eqdef{}& M~\textlog{cancellative} \land \knowInv{\iname}{\AuthInv(M, \pred, \gname)}
% \end{align*}

% The frame-preserving updates for $\auth{M}$ gives rise to the following view shifts:
% \begin{mathpar}
%  \inferH{NewAuth}
%   {\infinite(\mask) \and M~\textlog{cancellative}}
%   {\later\pred_\bot(a) \vs[\mask] \exists \iname \in \mask, \gname.\; \Auth(M, \pred, \gname, \iname) * \ownGhost{\gname}{\authfrag a : \auth{M}}}
%  \and
%  \axiomH{AuthOpen}
%   {\Auth(M, \pred, \gname, \iname) \vdash \ownGhost{\gname}{\authfrag \melt : \auth{M}} \vsE[\{\iname\}][\emptyset] \exists \melt_\f.\; \later\pred_\bot(\melt \mtimes \melt_\f) * \ownGhost{\gname}{\authfull \melt \mtimes \melt_\f, \authfrag a:\auth{M}}}
%  \and
%  \axiomH{AuthClose}
%   {\Auth(M, \pred, \gname, \iname) \vdash \later\pred_\bot(\meltB \mtimes \melt_\f) * \ownGhost{\gname}{\authfull a \mtimes \melt_\f, \authfrag a:\auth{M}} \vs[\emptyset][\{\iname\}] \ownGhost{\gname}{\authfrag \meltB : \auth{M}} }
% \end{mathpar}

% These view shifts in turn can be used to prove variants of the invariant rules:
% \begin{mathpar}
%  \inferH{Auth}
%   {\forall \melt_\f.\; \hoare{\later\pred_\bot(a \mtimes \melt_\f) * P}{\expr}{\Ret\val. \exists \meltB.\; \later\pred_\bot(\meltB\mtimes \melt_\f) * Q}[\mask]
%    \and \physatomic{\expr}}
%   {\Auth(M, \pred, \gname, \iname) \vdash \hoare{\ownGhost{\gname}{\authfrag a:\auth{M}} * P}{\expr}{\Ret\val. \exists \meltB.\; \ownGhost{\gname}{\authfrag \meltB:\auth{M}} * Q}[\mask \uplus \{\iname\}]}
%  \and
%  \inferH{VSAuth}
%   {\forall \melt_\f.\; \later\pred_\bot(a \mtimes \melt_\f) * P \vs[\mask_1][\mask_2] \exists \meltB.\; \later\pred_\bot(\meltB \mtimes \melt_\f) * Q(\meltB)}
%   {\Auth(M, \pred, \gname, \iname) \vdash
%    \ownGhost{\gname}{\authfrag a:\auth{M}} * P \vs[\mask_1 \uplus \{\iname\}][\mask_2 \uplus \{\iname\}]
%    \exists \meltB.\; \ownGhost{\gname}{\authfrag \meltB:\auth{M}} * Q(\meltB)}
% \end{mathpar}


% \subsection{Ghost heap}
% \label{sec:ghostheap}%
% FIXME use the finmap provided by the global ghost ownership, instead of adding our own
% We define a simple ghost heap with fractional permissions.
% Some modules require a few ghost names per module instance to properly manage ghost state, but would like to expose to clients a single logical name (avoiding clutter).
% In such cases we use these ghost heaps.

% We seek to implement the following interface:
% \newcommand{\GRefspecmaps}{\textsf{GMapsTo}}%
% \begin{align*}
%  \exists& {\fgmapsto[]} : \textsort{Val} \times \mathbb{Q}_{>} \times \textsort{Val} \ra \textsort{Prop}.\;\\
%   & \All x, q, v. x \fgmapsto[q] v \Ra x \fgmapsto[q] v \land q \in (0, 1] \\
%   &\forall x, q_1, q_2, v, w.\; x \fgmapsto[q_1] v * x \fgmapsto[q_2] w \Leftrightarrow x \fgmapsto[q_1 + q_2] v * v = w\\
%   & \forall v.\; \TRUE \vs[\emptyset] \exists x.\; x \fgmapsto[1] v \\
%   & \forall x, v, w.\; x \fgmapsto[1] v \vs[\emptyset] x \fgmapsto[1] w
% \end{align*}
% We write $x \fgmapsto v$ for $\exists q.\; x \fgmapsto[q] v$ and $x \gmapsto v$ for $x \fgmapsto[1] v$.
% Note that $x \fgmapsto v$ is duplicable but cannot be boxed (as it depends on resources); \ie we have $x \fgmapsto v \Lra x \fgmapsto v * x \fgmapsto v$ but not $x \fgmapsto v \Ra \always x \fgmapsto v$.

% To implement this interface, allocate an instance $\gname_G$ of $\FHeap(\Val)$ and define
% \[
% 	x \fgmapsto[q] v \eqdef
% 	  \begin{cases}
%     	\ownGhost{\gname_G}{x \mapsto (q, v)} & \text{if $q \in (0, 1]$} \\
%     	\FALSE & \text{otherwise}
%     \end{cases}
% \]
% The view shifts in the specification follow immediately from \ruleref{GhostUpd} and the frame-preserving updates in~\Sref{sec:fheapm}.
% The first implication is immediate from the definition.
% The second implication follows by case distinction on $q_1 + q_2 \in (0, 1]$.


%%% Local Variables:
%%% mode: latex
%%% TeX-master: "iris"
%%% End:

\endgroup\clearpage\begingroup
\section{Logical Paradoxes}
\newcommand{\starttoken}{\textsc{s}}
\newcommand{\finishtoken}{\textsc{f}}

In this section we provide proofs of some logical inconsistencies that arise when slight changes are made to the Iris logic.

\subsection{Saved Propositions without a Later}
\label{sec:saved-prop-no-later}

As a preparation for the proof about invariants in \Sref{app:section:invariants-without-a-later}, we show that omitting the later modality from a variant of \emph{saved propositions} leads to a contradiction.
Saved propositions have been introduced in prior work~\cite{dodds:higher-order-sync,iris2} to prove correctness of synchronization primitives; we will explain all that is necessary here.
The counterexample assumes a higher-order logic with separating conjunction, magic wand and the modalities $\always$ and $\upd$ satisfying the rules in \Sref{sec:base-logic}.

\begin{thm}
\label{thm:counterexample-1}
If there exists a type $\GName$ and a proposition $\_ \Mapsto \_ : \GName \to \Prop \to \Prop$ associating names $\gamma : \GName$ to propositions and satisfying:
\begin{align}
    \proves{}& \upd \Exists \gname : \GName. \gname \Mapsto P(\gname)
               \tagH{sprop-alloc} \\
    \gname \Mapsto P \proves{}& \always (\gname \Mapsto P)
               \tagH{sprop-persist} \\
    \gname \Mapsto \prop * \gname \Mapsto \propB \proves{}
             &
               \prop \Lra \propB
               \tagH{sprop-agree}
\end{align}
then $\proves\upd \FALSE$.
\end{thm}

The type $\GName$ should be thought of as the type of ``locations'' and $\gname \Mapsto P$ should be read as stating that location $\gname$ ``stores'' proposition $P$.
Notice that these are immutable locations, so the maps-to proposition is persistent.
The rule \ruleref{sprop-alloc} is then thought of as allocation, and the rule \ruleref{sprop-agree} states that a given location $\gname$ can only store \emph{one} proposition, so multiple witnesses covering the same location must agree.

%Compared to saved propositions in prior work, \ruleref{sprop-alloc} is stronger since the stored proposition can depend on the name being allocated.
%\derek{Can't we cut the above sentence?  This makes it sound like we are doing something weird that we ought not to be since prior work didn't do it.  But in fact, I thought that in our construction we do not really need to rely on this feature at all!  So I'm confused.}
The conclusion of \ruleref{sprop-agree} usually is guarded by a $\later$.
The point of this theorem is to show that said later is \emph{essential}, as removing it introduces inconsistency.
%
The key to proving \thmref{thm:counterexample-1} is the following proposition:
\begin{defn}
$A(\gname) \eqdef \Exists \prop : \Prop. \always\lnot \prop \land \gname \Mapsto \prop$.
\end{defn}
Intuitively, $A(\gname)$ says that the saved proposition named $\gname$ does \emph{not} hold, \ie we can disprove it.
Using \ruleref{sprop-persist}, it is immediate that $A(\gname)$ is persistent.

Now, by applying \ruleref{sprop-alloc} with $A$, we obtain a proof of $\prop \eqdef \gname \Mapsto A(\gname)$: this says that the proposition named $\gname$ is the proposition saying that it, itself, does not hold.
In other words, $\prop$ says that the proposition named $\gname$ expresses its own negation.
Unsurprisingly, that leads to a contradiction, as is shown in the following lemma:
\begin{lem}   \label{lem:saved-prop-counterexample-not-agname}   We have $\gname \Mapsto A(\gname) \proves \always\lnot A(\gname)$ and $\gname \Mapsto A(\gname) \proves A(\gname)$. \end{lem}
\begin{proof}%[\lemref{lem:saved-prop-counterexample-not-agname}]
\leavevmode
  \begin{itemize}
  \item First we show $\gname \Mapsto A(\gname) \proves \always\lnot A(\gname)$.
    Since $\gname \Mapsto A(\gname)$ is persistent it suffices to show $\gname \Mapsto A(\gname) \proves \lnot A(\gname)$.
    Suppose $\gname \Mapsto A(\gname)$ and $A(\gname)$.
    Then by definition of \(A\) there is a $\prop$ such that $\always \lnot \prop$ and $\gname \Mapsto \prop$.
    By \ruleref{sprop-agree} we have $\prop \Lra A(\gname)$ and so from $\lnot \prop$ we get $\lnot A(\gname)$, which leads to a contradiction with $A(\gname)$.
    
  \item Using the first item we can now prove $\gname \Mapsto A(\gname) \proves A(\gname)$.
    We need to prove
    \begin{align*}
      \Exists \prop : \Prop. \always \lnot \prop \land \gname \Mapsto \prop.
    \end{align*}
    We do so by picking $\prop$ to be $A(\gname)$, which leaves us to prove \(\always \lnot A(\gname) \land \gname \Mapsto A(\gname)\).
    The last conjunct holds by assumption, and the first conjunct follows from the previous item of this lemma.
  \end{itemize}
\end{proof}

With this lemma in hand, the proof of \thmref{thm:counterexample-1} is simple.
\begin{proof}[\thmref{thm:counterexample-1}]
  Using the previous lemmas we have
  \begin{align*}
    \proves \All \gname. \lnot (\gname \Mapsto A(\gname)).
  \end{align*}
  Together with the rule \ruleref{sprop-alloc} we thus derive $\upd \FALSE$.
\end{proof}

\subsection{Invariants without a Later}
\label{app:section:invariants-without-a-later}

Now we come to the main paradox: if we remove the $\later$ from \ruleref{inv-open}, the logic becomes inconsistent.
The theorem is stated as general as possible so taht it also applies to previous, less powerful versions of Iris.

\begin{thm}
  \label{thm:counterexample-2}
  Assume a higher-order separation logic with $\always$ and an update modality with a binary mask ${\upd}_{\set{0,1}}$ (think: empty mask and full mask) satisfying strong monad rules with respect to separating conjunction and such that:
  \begin{mathpar}
    \inferhref{weaken-mask}{eq:update-weaken-mask}
    {}{{\upd}_0 \prop \proves {\upd}_1 \prop}
  \end{mathpar}

\noindent
  Assume a type $\InvName$ and a proposition $\knowInv{\cdot}{\cdot} : \InvName \to \Prop \to \Prop$ satisfying:
%
  \begin{mathpar}
    \inferhref{inv-alloc}{eq:inv-alloc}
    {}
    {\prop \proves {\upd}_1 \Exists \iname. \knowInv \iname \prop}
    \and
    \inferhref{inv-persist}{eq:inv-persistent}
    {}
    {\knowInv \iname \prop \proves \always \knowInv \iname \prop}
    \and
    \inferhref{inv-open-nolater}{eq:inv-open}
    {\prop * \propB \proves {\upd}_0 (\prop * \propC) }
    {\knowInv \iname \prop * \propB \proves {\upd}_1 \propC}
  \end{mathpar}

\noindent
  Finally, assume the existence of a type $\GName$ and two tokens $\ownGhost{\cdot}{\starttoken} : \GName \to \Prop$ and $\ownGhost{\cdot}{\finishtoken}: \GName \to \Prop$ parameterized by $\GName$ and satisfying the following properties:
  \begin{mathpar}
    \inferhref{start-alloc}{eq:start-alloc}
    {}{\proves {\upd}_0 \Exists \gname. \ownGhost \gname \starttoken}
    \and
    \inferhref{start-finish}{eq:start-finish}
    {}{\ownGhost \gname \starttoken \proves {\upd}_0 \ownGhost \gname \finishtoken}
    \and
    \inferhref{start-not-finished}{eq:start-not-finished}
    {}{\ownGhost \gname \starttoken * \ownGhost \gname \finishtoken \proves \FALSE}
    \and
    \inferhref{finished-dup}{eq:finished-dup}
    {}{\ownGhost \gname \finishtoken \proves \ownGhost \gname \finishtoken * \ownGhost \gname \finishtoken}
  \end{mathpar}

\noindent
  Then $\TRUE \proves{\upd}_1 \FALSE$.
\end{thm}


The core of the proof is defining the $\Mapsto$ from the previous counterexample using invariants.
Then, using the standard proof rules for invariants, we show that it satisfies \ruleref{sprop-alloc} and \ruleref{sprop-persist}.
Furthermore, assuming the rule for opening invariants without a $\later$, we can prove a slightly weaker version of \ruleref{sprop-agree}, which is sufficient for deriving a contradiction.


% Taking ${\upd}_\bot$ and ${\upd}_\top$ to be the fancy update modalities $\pvs[\emptyset]$
% and $\pvs[\nat]$, respectively, we can see that Iris
% \emph{almost} satisfies these axioms.  First, to implement the tokens,
% we can use the RA with the carrier
% $\{\mundef,\epsilon,\starttoken,\finishtoken\}$ and operation
% $\epsilon \mtimes x = x \mtimes \epsilon = x$,
% $\finishtoken \mtimes \finishtoken = \finishtoken$ and otherwise
% $x \mtimes y = \mundef$.  Then, observe that the rules for
% $\knowInv{\cdot}{\cdot}$ are special cases of (derivable) invariant
% rules in Iris.  The fly in the ointment is the \ruleref{eq:inv-open}
% rule: in Iris, this rule would protect each occurrence of $\prop$
% in the premise of the rule with a $\later$, whereas here they are
% unprotected.

We start by defining $\Mapsto$ satisfying (almost) the assumptions of \lemref{lem:counterexample-invariants-saved-prop-agree}.
%
\begin{defn}
We define $\_ \Mapsto \_ : \GName \to \Prop \to \Prop$ as:
%
\begin{align*}
  \gname \Mapsto \prop \eqdef \Exists \iname. \knowInv \iname {\ownGhost \gname \starttoken \lor \ownGhost \gname \finishtoken * \always \prop}.
\end{align*}
\end{defn}
Note that using \ruleref{eq:inv-persistent}, it is immediate that $\gname \Mapsto \prop$ is persistent.

We use the tokens $\ownGhost \gname \starttoken$ and $\ownGhost \gname \finishtoken$ to model invariants that can be initialized ``lazily'': $\ownGhost \gname \starttoken$ indicates that the invariant is still not initialized, whereas the duplicable $\ownGhost \gname \finishtoken$ indicates it has been initialized with a resource satisfying $\prop$.%
%\footnote{We would usually require the token to be persistent, but it turns out the proof also works with the weaker assumption of duplicability.}
% RK: cut the footnote, it takes space. Maybe restore later

% TODO, explain this ...

We can show variants of \ruleref{sprop-agree} and \ruleref{sprop-alloc} for the defined $\Mapsto$.
\begin{lem}
  \label{lem:counterexample-invariants-saved-prop-alloc}
We have
  \(\proves {\upd}_\top \Exists \gname. \gname \Mapsto \prop(\gname)\).
\end{lem}
\begin{proof}
  We have to show the allocation rule \[\proves {\upd}_\top \Exists \gname. \gname \Mapsto \prop.\]
    From \ruleref{eq:start-alloc} we have a $\gname$ such that ${\upd}_\bot \ownGhost \gname \starttoken$ holds and hence from \ruleref{eq:update-weaken-mask} we have ${\upd}_\top\ownGhost\gname \starttoken$.
    Since we are proving a goal of the form ${\upd}_\top R$ we may assume $\ownGhost \gname \starttoken$.
    Thus for any $\prop$ we have ${\upd}_\top\left(\ownGhost{\gname}{\starttoken} \lor \ownGhost \gname \finishtoken * \prop\right)$.
    Again since our goal is still of the form ${\upd}_\top$ we may assume $\ownGhost{\gname}{\starttoken} \lor \ownGhost \gname \finishtoken * \always \prop$.
    The rule \ruleref{eq:inv-alloc} then gives us precisely what we need.
 \qed \end{proof}

%
\begin{lem}
\label{lem:counterexample-invariants-saved-prop-agree}
We have
  \(
  \gname \Mapsto \prop * \gname \Mapsto \propB * \always \prop \proves {\upd}_\top \always \propB
  \)
and thus
  \(
  \gname \Mapsto \prop * \gname \Mapsto \propB \proves ({\upd}_\top \always \prop) \Lra ({\upd}_\top \always \propB).
  \)
\end{lem}

\begin{proof}[\lemref{lem:counterexample-invariants-saved-prop-agree}]
\begin{itemize}
  \item We first show
    \[\gname \Mapsto \prop * \gname \Mapsto \propB * \always \prop \proves {\upd}_\top \always \propB.\]
    We use \ruleref{eq:inv-open} to open the invariant in $\gname \Mapsto \prop$ and consider two cases:
    % 
    \begin{enumerate}
    \item $\ownGhost \gname \starttoken$(the invariant is ``uninitialized'') : In this case, we use \ruleref{eq:start-finish} to ``initialize'' the invariant and obtain $\ownGhost{\gname}{\finishtoken}$.
      Then we duplicate $\ownGhost \gname \finishtoken$, and use it together with $\always \prop$ to close the invariant.
    \item $\ownGhost \gname \finishtoken * \always \prop$ (the invariant is ``initialized''): In this case we duplicate $\ownGhost \gname \finishtoken$, and use a copy to close the invariant.
    \end{enumerate}
    % 
    After closing the invariant, we have obtained $\ownGhost \gname \finishtoken$.
    Hence, it is sufficient to prove
    \[
      \ownGhost{\gname}{\finishtoken} * \gname \Mapsto \prop * \gname \Mapsto \propB * \always \prop \proves {\upd}_\top \always \propB.\]
    We proceed by using \ruleref{eq:inv-open} to open the other invariant in $\gname \Mapsto \propB$, and we again consider two cases:
    \begin{enumerate}
    \item $\ownGhost{\gname}{\starttoken}$ (the invariant is ``uninitialized''): As witnessed by \ruleref{eq:start-not-finished}, this cannot happen, so we derive a contradiction.
      Notice that this is a key point of the proof: because the two invariants ($\gname \Mapsto \prop$ and $\gname \Mapsto \propB$) \emph{share} the ghost name $\gname$, initializing one of them is enough to show that the other one has been initialized.
      Essentially, this is an indirect way of saying that really, we have been opening the same invariant two times.
    \item $\ownGhost{\gname}{\finishtoken} * \always \propB$ (the invariant is ``initialized''):
      Since $\always \propB$ is duplicable we use one copy to close the invariant, and retain another to prove ${\upd}_\top \always \propB$.
    \end{enumerate}
\item By applying the above twice, we easily obtain
\[ \gname \Mapsto \prop * \gname \Mapsto \propB \proves ({\upd}_\top \always \prop) \Lra ({\upd}_\top \always \propB) \]
\end{itemize}
\qed \end{proof}
% When allocating $\gname \Mapsto \prop(\gname)$ in \lemref{lem:counterexample-invariants-saved-prop-alloc}, we will start off in ``state'' $\ownGhost \gname \starttoken$, and once we have $P$ in \lemref{lem:counterexample-invariants-saved-prop-agree} we use \ruleref{eq:start-finish} to transition to $\ownGhost\gname \finishtoken$, obtaining ourselves a copy of said token.
% Finally, we use this token with $\gname \Mapsto \propB$ to obtain a proof of $\propB$.
Intuitively, \lemref{lem:counterexample-invariants-saved-prop-agree} shows that we can ``convert'' a proof from $\prop$ to $\propB$.

We are now in a position to replay the counterexample from \Sref{sec:saved-prop-no-later}.
The only difference is that because \lemref{lem:counterexample-invariants-saved-prop-agree} is slightly weaker than the rule \ruleref{sprop-agree} of \thmref{thm:counterexample-1}, we need to use ${\upd}_\top \FALSE$ in place of $\FALSE$ in the definition of the predicate $A$:
we let \(
  A(\gname) \eqdef \Exists \prop : \Prop. \always (\prop \Ra {\upd}_\top \FALSE) \land \gname \Mapsto \prop\)
and replay the proof that we have presented above.

%TODO: What about executing a view shift under a later?

%%% Local Variables:
%%% mode: latex
%%% TeX-master: "iris"
%%% End:

\endgroup\clearpage\begingroup
\printbibliography
\endgroup

\end{document}
